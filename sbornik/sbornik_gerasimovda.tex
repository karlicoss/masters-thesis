\documentclass{nsart_eng}
\usepackage{cite}
\usepackage{amsmath,amssymb}

% \usepackage{pscyr}
\usepackage[utf8]{inputenc}

\usepackage[russian]{babel}
\usepackage{graphicx}

\year{2016} \volume{0} \nomer{0} \firstpage{1}

\let\le\leqslant
\let\leq\leqslant
\let\ge\geqslant
\let\geq\geqslant


\usepackage{systeme}
\newcommand{\hilb}[1]{\mathcal{H}_{#1}}
\newcommand{\cconj}[1]{\overline{#1}}
\newcommand{\hank}[1]{H_{#1}^{(1)}}

\newcommand{\mcF}{\mathcal{F}}
\newcommand{\mcH}{\mathcal{H}} % ???
\newcommand{\mcI}{\mathcal{I}}
\newcommand{\mcL}{\mathcal{L}} % L2 space
\newcommand{\mcO}{\mathcal{O}} % Big O
\newcommand{\mcP}{\mathcal{P}}


\newcommand{\bbC}{\mathbb{C}} % complex plane
\newcommand{\bbD}{\mathbb{D}} % complex unit disk
\newcommand{\bbN}{\mathbb{N}}
\newcommand{\bbK}{\mathbb{K}}
\newcommand{\bbR}{\mathbb{R}}
\newcommand{\bbT}{\mathbb{T}} % complex unit circle
\newcommand{\bbZ}{\mathbb{Z}}

\newcommand{\eqdef}{\overset{\mathrm{def}}{=\joinrel=}}

\DeclareMathOperator{\dom}{dom}
\DeclareMathOperator{\ran}{Ran}
\DeclareMathOperator{\rng}{rng}

\newcommand{\todo}[1]{\textcolor{red}{{\large TODO: #1}}}

\newenvironment{elist}{\begin{easylist}[enumerate]}{\end{easylist}}
\newenvironment{ilist}{\begin{easylist}[itemize]}{\end{easylist}}

\newcommand{\myspecial}[1]{\mathrm{#1}}

% imaginary unit
\newcommand{\iu}{{i\mkern1mu}}


\newcommand{\ipcdot}{\ip{\cdot}{\cdot}}
\newcommand{\iip}[2]{[#1,#2]}
\newcommand{\iipcdot}{\iip{\cdot}{\cdot}}

\newcommand{\dsum}{\oplus}
\newcommand{\ddiff}{\ominus}
% indefinite direct sum
\newcommand{\idsum}{[+]}
\newcommand{\iddiff}{[-]}

\DeclarePairedDelimiter{\Vector}{\lparen}{\rparen}

\newcommand{\tit}{\textit}
\newcommand{\cls}{\overline}
\newcommand{\eps}{\varepsilon}


\newcommand{\argmin}{\operatornamewithlimits{argmin}}
\newcommand{\argmax}{\operatornamewithlimits{argmax}}

\renewcommand{\Re}{\operatorname{Re}}
\renewcommand{\Im}{\operatorname{Im}}
\renewcommand{\phi}{\varphi} % TODO is there a prettier way to do that

\newcommand{\eexp}[1]{e^{#1}}

\DeclareMathOperator\atanh{atanh}

% ???
% \newcommand{\abs}[1]{\left| #1 \right|}
% \newcommand{\norm}[1]{\left\lVert #1 \right\rVert}
\newcommand*\Eval[3]{\left.#1\right\rvert_{#2}^{#3}}
\usepackage{mathtools} % xmapsto, \Vector

\DeclarePairedDelimiter{\abs}{\lvert}{\rvert}
\newcommand{\eexp}[1]{e^{#1}}
\newcommand{\iu}{{i\mkern1mu}}
\renewcommand{\Re}{\operatorname{Re}}
\renewcommand{\Im}{\operatorname{Im}}

\begin{document}

\title[короткое название статьи]
{НАЗВАНИЕ СТАТЬИ}

\author[А.\,Б.~Иванов, C.\,C.~Claus, В.\,Г.~Петров]
{$^1$А.\,Б.~Иванов, $^{1}$C.\,C.~Claus, $^2$В.\,Г.~Петров}

\address{
$^1$ Санкт-Петербургский Национальный Исследовательский Университет Информационных Технологий, \\
Механики и Оптики,\\
Кронверкский пр., 49, Санкт-Петербург, 197101, Россия\\
$^2$ Swiss Federal University of Technology,\\
Sonneggstrasse, 5,  Zurich, CH-8092, Switzerland }

\email{ivanov@ivanov.ru, claus@claus.ch, petrov@petrov.ru }

УДК ???.??, ???.???.?% insert UDK

\begin{abstract}
Рассматривается задача рассеивания на квантовом графе $\Gamma$, представляющем из себя кольцо $\Omega$, связянное с каналом (TODO?) дельта-граничным условием???, параметризованным вещественной константой $a$. Изучается поведение системы при различных $a$, и полнота резонансных состояний оператора Шредингера графа $\Gamma$ в пространстве $\mcL_2(\Omega)$.
\end{abstract}

\keywords{задача рассеяния, квантовый граф, резонансы, уравнение Шредингера}

\maketitle

\section{Введение}

Квантовый граф — широко используемая модель наносистемы. Если граф $\Gamma$ состоит только из конечного числа ребер конечной длины, его гамильтониан имеет чисто дискретный спектр, и его собственные функции образуют полную систему в $\mcL_2(\Gamma)$. Если же граф $\Gamma$ представляет из себя резонатор $\Omega$ с полубесконечными ребрами, в спектре будет присутствовать непрерывная часть и резонансы, индуцированные собственными числами гамильтониана резонатора $\mcH_\Omega$. Резонансные состояния оператора Шредингера не принадлежат пространству $\mcL_2(\Gamma)$, однако, при сужении их на конечный домен $\Omega$, становятся квадратично интегрируемыми и лежат в пространстве $\mcL_2(\Omega)$. Для многих приложений (TODO каких? ссылки?) интересно знать, формируют ли полную систему резонанстные состояния графа $\Gamma$ в пространстве $\mcL_2(\Omega)$ резонатора.

\section{Цель работы и постановка задачи}
В статье TODO ссылка была показана полнота резонансных функций на графе с резонатором, состоящим из отрезка с $\delta$-граничным условием. Естественным предположением является что произвольный резонатор состоящий из конечного числа ребер конечной длины должен обладать похожими свойствами, однако, это не так. В этой работе исследуется резонатор, состоящий из кольца и показывается неполнота резонансных состояний на этом кольце.

\section{TODO ???? }
В общем виде волновые функции слева и справа от резонатора имеют вид:
\[
\psi_L(x) = A \eexp{\iu k x} + B \eexp{-\iu k x}
\]
\[
\psi_R(x) = C \eexp{\iu k x} + D \eexp{-\iu k x}
\]

S-матрица выражает зависимость исходящей волны от входящей:
\[
\begin{pmatrix} B \\ C \end{pmatrix} = S \begin{pmatrix} A \\ D \end{pmatrix}
\]

В случае конечной размерности вспомогательного пространства есть простой критерий отсутствия сингулярного множителя (TODO ссылку):

Для скалярных внутренних функций $\phi$ существует критерий отстутствия сингулярного множителя: TODO ссылка Nikolskii, p.99
\[
\lim\limits_{r = 1} \int\limits_{\bbT} \log \abs{\phi(r \zeta)} d m(\zeta) = 0
\]

Так как $\det S$ является скалярной внутренней функцией, мы можем воспользоваться данным критерием для исследования полноты системы резонантных состояний.
% @book{peetre2012treatise,
%   title={Treatise on the Shift Operator: Spectral Function Theory},
%   author={Peetre, J. and Nikol'skii, N.K. and Hruscev, S.V. and Peller, V.V.},
%   isbn={9783642701511},
%   lccn={84026869},
%   series={Grundlehren der mathematischen Wissenschaften},
%   url={https://books.google.ru/books?id=SlHoCAAAQBAJ},
%   year={2012},
%   publisher={Springer Berlin Heidelberg}
% }
% $\det S$ — произведение Бляшке-Потапова $\iff$ TODO ??? 


\section{Модель исследования}
Рисунок: квантовый граф $\Gamma$, состоящий из полубесконечных ребер $\Omega_L, \Omega_R$ и рассеивателя $\Omega$, представляющего из себя окружность длиной $1$.

Мы рассматриваем случай рассеяния волны c волновым вектором $k$, приходящей слева направо. Таким образом, волновые функции на различных частях графа принимают следующий вид:

\begin{align*}
\psi_L(x) &= \eexp{\iu k x} + R \eexp{-\iu k x} \\
\psi_R(x) &= T \eexp{\iu k x}\\
\psi_\Omega(x) &= P \sin(k x) + Q \cos(k x)
\end{align*}
, где $R$ и $T$ — коэффициенты отражения и прохождения волны. Так как система симметрична, ее матрица рассеяния принимает вид
$S(k) = \begin{pmatrix} R(k) & T(k) \\ T(k) & R(k) \end{pmatrix}$

В вершине $V$ мы ставим граничное условие Дирихле на волновую функцию, и $\delta$-условие со связью $a$ на ее производную:

% TODO SYSTEM
\begin{align*}
\psi_L(0) = \psi_R(0) = \psi_\Omega(0) = \psi_\Omega(1) \\ 
-\psi'_L(0) + \psi'_\Omega(0) - \psi'_\Omega(1) + \psi'_R(0) = a \psi_L(0)
\end{align*}


\section{Вычисление S-матрицы}

Решаем систему:
\begin{align*}
& 1 + R &= T \\
& 1 + R &= Q \\
& Q \cos k + P \sin k &= T \\
& -P k \cos k + Q k \sin k + P k + \iu R k + \iu T k - \iu k &= T a
\end{align*}

Решая систему, получаем:

\begin{align*}
R(k) = -\frac{2 \, k \cos\left(k\right) + a \sin\left(k\right) - 2 \, k}{2 \, k \cos\left(k\right) + {\left(a - 2 i \, k\right)} \sin\left(k\right) - 2 \, k} \\
T(k) = -\frac{2 i \, k \sin\left(k\right)}{2 \, k \cos\left(k\right) + {\left(a - 2 i \, k\right)} \sin\left(k\right) - 2 \, k}
\end{align*}
, подставляя полученные значения коэффициента прохождения и отражения в S-матрицу, получаем:
\[
\det S = 
\frac
{\cos\left(k\right) + {\left(\frac{a}{2 k} + i\right)} \sin\left(k\right) - 1}
{\cos\left(k\right) + {\left(\frac{a}{2 k} - i\right)} \sin\left(k\right) - 1}
\]


\section{Исследование полноты при $a=0$}
\[
\det S
= \frac
{\cos\left(k\right) + \iu \sin\left(k\right) - 1}
{\cos\left(k\right) - \iu \sin\left(k\right) - 1}
= \frac{\eexp{\iu k} - 1}{\eexp{-\iu k} - 1}
= -\eexp{i k}
\]

\[
\ln \abs{\det S} = \ln \eexp{- \Im k} = -\Im k
\]

% Преобразование Кэли: $\zeta = \frac{k - i}{k + i}$, обратное: $k = \iu \frac{1 + \zeta}{1 - \zeta}$.

Вычислим интеграл в пространстве единичного диска, для этого применим обратное преобразование Кэли $k \to \iu \frac{1 + \zeta}{1 - \zeta}$ к подынтегральной функции: $\Im k \to \Im \left( \iu \frac{1 + \zeta}{1 - \zeta} \right) $.

%  TODO вычислим интеграл под пределом в зависимости от R:
% TODO varphi??
\[
  \lim\limits_{R = 1} \int\limits_{\abs{\zeta} = R} \ln \abs{\det S(\zeta)} d \zeta
= \lim\limits_{R = 1} \int\limits_{\abs{\zeta} = R} \Im \left( \iu \frac{1 + \zeta}{1 - \zeta} \right)  d\zeta = \dots
\]

Перейдем в полярные координаты: $\zeta \to R \eexp{\iu \phi}, d\zeta \to R \iu \eexp{\iu \phi}$:
\[
\dots = \lim\limits_{R = 1} \int\limits_{\abs{\zeta} = R} \Im \left( \iu \frac{1 + R \eexp{\iu \phi}}{1 - R \eexp{\iu \phi}} \right) R \iu \eexp{\iu \phi} d\phi
\]

% TODO proper d

Комплексный интеграл складывается из сумм интегралов действительной и мнимной части подынтегрального выражения. Рассчитаем мнимую часть:

\begin{align*}
\Im \left(  \Im \left( \iu \frac{1 + R \eexp{\iu \phi}}{1 - R \eexp{\iu \phi}} \right) R \iu \eexp{\iu \phi} \right)
 &= R \Re \left(  \Re \left( \frac{1 + R \eexp{\iu \phi}}{1 - R \eexp{\iu \phi}} \right) \eexp{\iu \phi} \right) \\
 &= R \Re \left( \frac{1 + R \eexp{\iu \phi}}{1 - R \eexp{\iu \phi}} \right) \Re \left(   \eexp{\iu \phi} \right) \\
 &= R \Re \left( \frac{(1 + R \eexp{\iu \phi}) (1 - R \eexp{-\iu \phi}) }{(1 - R \eexp{\iu \phi}) (1 - R \eexp{-\iu \phi})} \right) \cos \phi \\
 &= R \Re \left( \frac{1 - R^2 + 2 \iu R \sin \phi}{1 + R^2 - 2 R \cos \phi} \right) \cos \phi \\
 &= R \frac{1 - R^2}{1 + R^2 - 2 R \cos \phi} \cos \phi \\
 % TODO calculate the integral properly
 &= 2 \pi R^2
\end{align*}

Можно видеть, что предел мнимой части при $R \to 1$ равен $2 \pi$, следовательно, по критерию полноты, система резонатных состояний графа $\Gamma$ не является полной на кольце $\Omega$.

Ссылки на литературу в тексте делаются следующим образом
\cite{1,4}.


Формула с нумерацией:
\begin{equation} \label{eq1}
\alpha (t) = \frac{ \beta (t)}{\gamma (t)}.
\end{equation}

Многострочные формулы можно задавать так:
\begin{multline} \label{eq2}
\sum\limits_{n=0}^\infty  {\frac{{f^{\left( n \right)} \left( {x_0 } \right)}}{{n!}}\left( {x - x_0 } \right)^n }  = \\
= f\left( {x_0 } \right) + \frac{{f'\left( {x_0 }
\right)}}{{1!}}\left( {x - x_0 } \right) +
 \frac{{f''\left( {x_0 } \right)}}{{2!}}\left( {x - x_0 } \right)^2 + \dots +
 \frac{{f^{\left( n \right)} \left( {x_0 } \right)}}{{n!}}\left( {x - x_0 } \right)^n  + \dots
\end{multline}

Ссылки на формулы задаются так  (\ref{eq1}).

Рисунок вставляется следующим образом:


\begin{figure}
\includegraphics{fig1.png}
\caption{Черно-белая картинка}
\end{figure}


\subsection{Название подпараграфа 2}
Текст подпараграфа 2.

\section{Заключение}
Текст заключения

% \section*{Благодарности}
% Текст.

\begin{thebibliography}{99}
\bibitem{1}
Автор~А.\,А., Автор~Б.\,Б. {\it Название книги}.  Издательство,  город, 281~с. (2000)

\bibitem{2}
Автор~В.\,В., Автор~Г.\,Г. Название статьи. {\it Название журнала}, {\bf 1} (5), С 1 -- 3. (2000)

\bibitem{3}
Автор~Д.\,Д., Автор~Е.\,Е. Название доклада. Сборник трудов конференции "Конференция", место и дата, С. 47 -- 49

\bibitem{3}
Автор~Ж.\,Ж., Автор~З.\,З. Название статьи.  2010.
URL/arXiv: http://books.ifmo.ru/ntv.

\bibitem{4}
Название патента: патент 1111111 Россия MMM H04 B 1/36, Иванов И.И., владелец патента. OGOGO, N 2000131517/09, Bull. N 12, 3 с.


\end{thebibliography}

\end{document}
