\documentclass[12pt, a4paper]{article}

\usepackage{cite}
\usepackage{amsmath,amssymb}

\usepackage[utf8]{inputenc}

\usepackage{graphicx}

\usepackage{mathtools} % \DeclarePairedDelimiter
\usepackage{xcolor} % \textcolor

% \DeclarePairedDelimiter{\abs}{\lvert}{\rvert}
\newcommand{\abs}[1]{\left| #1 \right|}
\newcommand{\norm}[1]{\left\lVert #1 \right\rVert}
\newcommand*\Eval[3]{\left.#1\right\rvert_{#2}^{#3}}
% \DeclarePairedDelimiterX{\norm}[1]{\lVert}{\rVert}{#1}

\newcommand{\mcL}{\mathcal{L}}
\newcommand{\mcH}{\mathcal{H}}
\newcommand{\bbT}{\mathbb{T}}
\newcommand{\bbC}{\mathbb{C}}
\newcommand{\bbD}{\mathbb{D}}

\newcommand{\eexp}[1]{e^{#1}}
\newcommand{\iu}{{i\mkern1mu}}
\renewcommand{\Re}{\operatorname{Re}}
\renewcommand{\Im}{\operatorname{Im}}
\renewcommand{\phi}{\varphi}
\newcommand{\todo}[1]{{\large \textcolor{red}{TODO: #1}}}

\DeclareMathOperator\atanh{atanh}
\DeclareMathOperator\atan{atan}

\begin{document}

Scatterer consists of $W$ wires of length $1$, on the left joint vertex $\delta$-well with depth $a$, on the right: $\delta$-well with depth $b$. 

\todo{Picture}

\todo{System of equations}

\todo{Solutions for transmission/refletion?}

Simplifying, we get:

\[
\det S(k) = \frac{W {\left(a + b\right)} k \cos\left(k\right) + 2 i \, W k^{2} \cos\left(k\right) + {\left(i \, a + i \, b\right)} k \sin\left(k\right) - {\left({\left(W^{2} + 1\right)} k^{2} - a b\right)} \sin\left(k\right)}{W {\left(a + b\right)} k \cos\left(k\right) - 2 i \, W k^{2} \cos\left(k\right) - {\left(i \, a + i \, b\right)} k \sin\left(k\right) - {\left({\left(W^{2} + 1\right)} k^{2} - a b\right)} \sin\left(k\right)}
\]



\section*{Proof of completeness for $a = b = 0$}
Under the substitituion $a = 0, b = 0$, $\det S$ simplifies to:

\[
\det S(k) = - \frac{2 i \, W \cos\left(k\right) - {\left(W^{2} + 1\right)} \sin\left(k\right)}{2 i \, W \cos\left(k\right) + {\left(W^{2} + 1\right)} \sin\left(k\right)}
\]

Also, for the sake of brevity, we define 

\[
s(k) = \abs{\det S(k)}
\]

, so we seek the proof of convergence of integral \todo{blah blah $s(k)$}

% We'll use the inequality:

% \[
% \frac{x}{1 + x} \le \ln(1 + x)
% \]
% , which holds for $-1 < x \le 0$. % As $x$ goes to $-1$, 

% We'll rewrite it in a more convenient form by a substitution $x \to s(k) - 1$:

% \[
% 1 - \frac{1}{s(k)} \le \ln s(k)
% \]

% Since $s(k)$ 

\subsection{Upper bound for integral}

We know that $s(k)$ is bounded by $1$ from above in the upper complex plane  ($\Im k \ge 0$) \todo{why?}, hence $\ln \abs{\det S} \le 0$ and the desired integral is bounded from above by $0$. However, we have to show that the integral is bounded from below as well. So, next, we will compute a lower bound $l(k)$ for $s(k)$, such that $0 \le l(k) \le s(k)$, and will prove that $\int \ln l(k)$ converges to $0$. Since $\ln l(k) \le \ln s(k)$, that will prove convergence of the original integral.

\subsection{Notice that $s(k)$ is periodic w.r.t. real axis}
\todo{Plot of $\abs{\det S}$ ?}

% Since $\sin k$ and $\cos k$ are periodic w.r.t the real axis, with the period of $2 \pi$, 
\todo{is this step really necessary?}

Notice that all zeroes have $\Im k = \atanh \frac{2 W}{W^2 + 1}$. For brevity, let's define $Z = \atanh \frac{2 W}{W^2 + 1}$.

\subsection{Notice that for any $x$, $s(x + \iu y) \le s(\iu y)$}
\todo{How to prove? We could probably take a look at numerator and denominator in separate?}

Now, we are free to replace $s(x + \iu y)$ with:

\[
s_1(x + \iu y)
 = \abs{\frac{(W^2 + 1) \sinh y - 2 W \cosh y}{(W^2 + 1) \sinh y + 2 W \cosh y}}
\]
, now that $\cosh y \ge 1$, we are free to divide by $\cosh y$:

\[
s_1(x + \iu y)
 = \abs{\frac{\tanh y - \frac{2 W}{W^2 + 1}}{\tanh y + \frac{2 W}{W^2 + 1}}}
\]

Note that now $s_1(k)$ has zeroes along the line $\Im k = Z = \atanh \frac{2 W}{W^2 + 1}$, which means $\ln s_1(k)$ will have continuous number of singularities along this line, whereas the original function $\ln\abs{\det S(k)}$ only had countable number of singularities. However, we'll show that the function is still integrable.

\todo{Plot?}

Next, since $\tanh y \ge 0$ for $y \ge 0$ and is strictly increasing, we can get rid of the absolute value:

\todo{make the formula bigger}
\[
s_1(x + \iu y)
 = \begin{cases}
 \frac{\frac{2 W}{W^2 + 1} - \tanh y}{\tanh y + \frac{2 W}{W^2 + 1}}, & 0 \le y \le Z \\
 \frac{\tanh y - \frac{2 W}{W^2 + 1}}{\tanh y + \frac{2 W}{W^2 + 1}}, & y > Z 
 \end{cases}
\]

Now, the estimation splits in two cases:

\subsection*{Case 1: $0 \le y \le Z$}
Since denominator $\tanh y + \frac{2 W}{W^2 + 1}$ only increases as $0 \le y \le Z$, we can replace it with its maximum value $\tanh Z + \frac{2 W}{W^2 + 1} = \frac{4 W}{W^2 + 1}$, thus getting estimate $s_2$:

\[
s_2(x + \iu y) = \frac{\frac{2 W}{W^2 + 1} - \tanh y}{\frac{4 W}{W^2 + 1}} = \frac{1}{2} - \frac{W^2 + 1}{4 W} \tanh y \le s_1(x + \iu y)
\]

Since $\tanh(y)$ is concave, $s_2$ is convex \todo{proof} on $0 \le y \le Z$, we can approximate it from below using its first derivative at $Z$:

\[
s_3(x + \iu y) = s_2'(Z)(y - Z) = \frac{{\left(W^{2} - 1\right)}^{2}}{4 \, {\left(W^{2} + 1\right)} W} (Z - y)
\]
% \[
% h'(Z) = \frac{{\left(W^{2} - 1\right)}^{2}}{4 \, {\left(W^{2} + 1\right)} W}
% \]
% h'(W=2) = 9/40; h'(W=3) = 8/15

Since $s_3$ is stricly decreasing, $s(0) = \frac{{\left(W^{2} - 1\right)}^{2}}{4 \, {\left(W^{2} + 1\right)} W} Z$ is strictly positive and $s(Z) = 0$, $s_3(k)$ is non-negative and $\ln s_3(k)$ is well defined on $0 \le y \le Z$.

\todo{We might need one more estimate to ensure $l(x + 0 \iu) = 1$}


\subsection*{Case 2: $y > Z$}
For $y > Z$, 
\[
s_1(x + \iu y) 
 = \frac{\tanh y - \frac{2 W}{W^2 + 1}}{\tanh y + \frac{2 W}{W^2 + 1}}
 = 1 - 2 \frac{\frac{2 W}{W^2 + 1}}{\tanh y + \frac{2 W}{W^2 + 1}}
\]

% TODO
% https://proofwiki.org/wiki/Inverse_of_Strictly_Increasing_Strictly_Concave_Real_Function_is_Strictly_Convex
Since $s_1$ is concave \todo{proof? just take a look at second derivative?}, and strictly increasing, we can take any $Z_0 > Z$ and approximate the function by a piecewise linear function $s_4$.

\[
s_4(x + \iu y) = 
\begin{cases}
\frac{s_1(Z_0)}{Z_0 - Z} (y - Z) &, Z < y \le Z_0 \\
s_1(Z_0) &, y > Z_0
\end{cases}
\]

For simplicity, take $Z_0 = Z + 1$, then:

\[
s_4(x + \iu y) = 
\begin{cases}
s_1(Z + 1) (y - Z) &, Z < y \le Z + 1 \\
s_1(Z + 1) &, y > Z + 1
\end{cases}
\]


\subsection*{Proof of convergence for constant part}
\todo{Might be easier in Cayley space}

\subsection*{Proofs of convergence for logarithmic parts}

\todo{For second, we could probably get rid of $\frac{1}{(k-1)^2}$ via the minimum estimation}

\todo{For first, things are slightly harder. Prove about generic function with zero at $Z$, then first implies second???}
Peak induced by $k=1$ is quite nasty. Ensure it converges, implement a checker. Although, it might not, it probably gets smoothed by $\abs{\det S} = 1$ for the original function....


\todo{Unit cicle in Cayley space maps to $R$, and $\ln\abs{\det S}$ on $R$ is zero everywhere}

\subsection*{Proof of convergence}

\[
l(k) =
\begin{cases}
1 \\
2 \\
3 
\end{cases}
\]

Not than we have an estimate for $s(k)$ on the whole upper complex plane, we can compute the lower bound for the original integral. For fixed $r$ (radius of the contour in Cayley space), the integral looks like:

\[
\int\limits_{C_r} \ln l(k) \frac{1}{(k + i)^2} dk
\]

, where $C_r$ is the image of the curve $\abs{\zeta} = r$ under the inverse Cayley transform.

Cayley transform: $W(z) = \frac{z - \iu}{z + \iu}$, inverse: $w(\zeta) = \iu \frac{1 + \zeta}{1 - \zeta}$.

$C_r$ is the circle with radius $R(r) = \Im \frac{w(r) - w(-r)}{2}$ and center $C(r) = \Im \frac{w(r) + w(-r)}{2}$.

Substituting $k \to R \eexp{\iu t} + \iu C$, we get:

\todo{multiply by $2 \iu$ due to derivative of Cayley transform}

\[
\int\limits_{0}^{2 \pi} \ln l(R \eexp{\iu t} + \iu C) \frac{1}{(R \eexp{\iu t} + \iu C + i)^2} R \iu dt
\]

To estimate this integral, we'll use Cauchy-Schwartz inequality:

\[
\big| \langle u,v \rangle \big| \leq \left\|u\right\| \left\|v\right\|
\]

, or, to be more specific, its $\mcL^2[a, b]$ version:

\[
\abs{
\int\limits_{t=a}^{b} f(t) g^*(t) dt
}^2
\le
\int\limits_{t=a}^b \abs{f(t)}^2 dt 
\int\limits_{t=a}^b \abs{g(t)}^2 dt 
\]

To make use of the inequality, we take
\begin{align*}
a      &= 0 \\
b      &= 2 \pi \\
f(t)   &= \ln l(R \eexp{\iu t} + \iu C) \frac{\sqrt{R}}{R \eexp{\iu t} + \iu C + i} \\
g^*(t) &= \iu \frac{\sqrt{R}}{R \eexp{\iu t} + \iu C + i}
\end{align*}

\subsection*{Estimate of $\int \abs{g(t)}^2$}

\begin{align*}
\abs{g(t)}^2 = \abs{g^*(t)}^2
&=   \frac{\sqrt{R}^2}{\abs{R \cos t + \iu R \sin t + \iu C + i}^2} \\
&=   \frac{R}{R^2 \cos^2 t + (R \sin t + C + 1)^2} \\
&= \frac{R}{R^2 \cos^2 t + R^2 \sin^2 t + (C + 1)^2  + 2 R (C + 1) \sin t} \\
&=   \frac{1}{R + (C + 1) \frac{C + 1}{R} + 2 (C + 1) \sin t} \\ 
\end{align*}

If we take $a = R + (C + 1) \frac{C + 1}{R}$, $b = 2 (C + 1)$, and notice $a > b$ \todo{Prove}, the integral is well known \todo{reference}:
\[
\int \frac{dx}{a + b \sin x} = \frac{2}{\sqrt{a^2 - b^2}} \atan \frac{a \tan \frac{x}{2} + b}{\sqrt{a^2 - b^2}}
\]

% Notice that since 

\begin{align*}
a^2 - b^2
& =  (R + (C + 1) \frac{C + 1}{R})^2 - (2 (C + 1))^2\\
& =  R^2 + \frac{(C+1)^4}{R^2} + 2 (C+1)^2 - 4 (C + 1)^2 \\
& =  (R - (C + 1) \frac{C + 1}{R})^2 && \text{,and, since $C \ge R$} \\
&\ge (R - (R + 1) \frac{R + 1}{R})^2 \\
&\ge (2 + \frac{1}{R})^2 \\
&\ge 4
\end{align*}

We can choose the branch cut of the antiderivative in such a way that integration from $0$ to $2 \pi$ only affects a single Riemann sheet, therefore, we can use the fact that $\abs{\int\limits_a^b f(x) dx} \le \abs{F(b) - F(a)} \le \abs{F(a)} + \abs{F(b)}$.

Now, since $\frac{2}{\sqrt{a^2 - b^2}} \le \frac{2}{2} = 1$ and $\atan$ is bounded by $\frac{\pi}{2}$, by direct application of the fact above, we conclude that than $\pi$.


\end{document}