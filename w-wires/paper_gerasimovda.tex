\documentclass[12pt, a4paper]{article}

\usepackage{cite}
\usepackage{amsmath,amssymb}

\usepackage[utf8]{inputenc}

\usepackage{graphicx}

\usepackage{mathtools} % \DeclarePairedDelimiter
\usepackage{xcolor} % \textcolor

\DeclarePairedDelimiter{\abs}{\lvert}{\rvert}

\newcommand{\mcL}{\mathcal{L}}
\newcommand{\mcH}{\mathcal{H}}
\newcommand{\bbT}{\mathbb{T}}
\newcommand{\bbC}{\mathbb{C}}
\newcommand{\bbD}{\mathbb{D}}

\newcommand{\eexp}[1]{e^{#1}}
\newcommand{\iu}{{i\mkern1mu}}
\renewcommand{\Re}{\operatorname{Re}}
\renewcommand{\Im}{\operatorname{Im}}
\renewcommand{\phi}{\varphi}
\newcommand{\todo}[1]{{\large \textcolor{red}{TODO: #1}}}

\DeclareMathOperator\atanh{atanh}

\begin{document}

Scatterer consists of $W$ wires of length $1$, on the left joint vertex $\delta$-well with depth $a$, on the right: $\delta$-well with depth $b$. 

\todo{Picture}

\todo{System of equations}

\todo{Solutions for transmission/refletion?}

Simplifying, we get:

\[
\det S(k) = \frac{W {\left(a + b\right)} k \cos\left(k\right) + 2 i \, W k^{2} \cos\left(k\right) + {\left(i \, a + i \, b\right)} k \sin\left(k\right) - {\left({\left(W^{2} + 1\right)} k^{2} - a b\right)} \sin\left(k\right)}{W {\left(a + b\right)} k \cos\left(k\right) - 2 i \, W k^{2} \cos\left(k\right) - {\left(i \, a + i \, b\right)} k \sin\left(k\right) - {\left({\left(W^{2} + 1\right)} k^{2} - a b\right)} \sin\left(k\right)}
\]



\section*{Proof of completeness for $a = b = 0$}
Under the substitituion $a = 0, b = 0$, $\det S$ simplifies to:

\[
\det S(k) = - \frac{2 i \, W \cos\left(k\right) - {\left(W^{2} + 1\right)} \sin\left(k\right)}{2 i \, W \cos\left(k\right) + {\left(W^{2} + 1\right)} \sin\left(k\right)}
\]

Also, for the sake of brevity, we define 

\[
s(k) = \abs{\det S(k)}
\]

, so we seek the proof of convergence of integral \todo{blah blah $s(k)$}

We'll use the inequality:

\[
\frac{x}{1 + x} \le \ln(1 + x)
\]
, which holds for $-1 < x \le 0$. % As $x$ goes to $-1$, 

We'll rewrite it in a more convenient form by a substitution $x \to s(k) - 1$:

\[
1 - \frac{1}{s(k)} \le \ln s(k)
\]

Since $s(k)$ 

\subsection{Upper bound for integral}

We know that $s(k)$ is bounded by $1$ from above in the upper complex plane  ($\Im k \ge 0$) \todo{why?}, hence $\ln \abs{\det S} \le 0$.

Outline of the following steps: we compute a lower bound $l(k)$ for $s(k)$, such that $0 \le l(k) \le s(k)$, and we prove that $\int \ln l(k)$ converges. Since $\ln l(k) \le \ln s(k)$, that will prove convergence of the original integral.

\subsection{Notice that $s(k)$ is periodic w.r.t. real axis}
\todo{Plot of $\abs{\det S}$ ?}

% Since $\sin k$ and $\cos k$ are periodic w.r.t the real axis, with the period of $2 \pi$, 
\todo{is this step really necessary?}

Notice that all zeroes have $\Im k = \atanh \frac{2 W}{W^2 + 1}$. For brevity, let's define $Z = \atanh \frac{2 W}{W^2 + 1}$.

\subsection{Notice that for any $x$, $s(x + \iu y) \le s(\iu y)$}
\todo{How to prove? We could probably take a look at numerator and denominator in separate?}

Now, we are free to replace $s(x + \iu y)$ with:

\[
s_1(x + \iu y)
 = \abs{\frac{(W^2 + 1) \sinh y - 2 W \cosh y}{(W^2 + 1) \sinh y + 2 W \cosh y}}
\]
, now that $\cosh y \ge 1$, we are free to divide by $\cosh y$:

\[
s_1(x + \iu y)
 = \abs{\frac{\tanh y - \frac{2 W}{W^2 + 1}}{\tanh y + \frac{2 W}{W^2 + 1}}}
\]

Note that now $s_1(k)$ has zeroes along the line $\Im k = Z = \atanh \frac{2 W}{W^2 + 1}$, which means $\ln s_1(k)$ will have continuous number of singularities along this line, whereas the original function $\ln\abs{\det S(k)}$ only had countable number of singularities. However, we'll show that the function is still integrable.

\todo{Plot?}

Next, since $\tanh y \ge 0$ for $y \ge 0$ and is strictly increasing, we can get rid of the absolute value:

\todo{make the formula bigger}
\[
s_1(x + \iu y)
 = \begin{cases}
 \frac{\frac{2 W}{W^2 + 1} - \tanh y}{\tanh y + \frac{2 W}{W^2 + 1}}, & 0 \le y \le Z \\
 \frac{\tanh y - \frac{2 W}{W^2 + 1}}{\tanh y + \frac{2 W}{W^2 + 1}}, & y > Z 
 \end{cases}
\]

Now, the estimation splits in two cases:

\subsection*{Case 1: $0 \le y \le Z$}
Since denominator $\tanh y + \frac{2 W}{W^2 + 1}$ only increases as $0 \le y \le Z$, we can replace it with its maximum value $\tanh Z + \frac{2 W}{W^2 + 1} = \frac{4 W}{W^2 + 1}$, thus getting estimate $s_2$:

\[
s_2(x + \iu y) = \frac{\frac{2 W}{W^2 + 1} - \tanh y}{\frac{4 W}{W^2 + 1}} = \frac{1}{2} - \frac{W^2 + 1}{4 W} \tanh y \le s_1(x + \iu y)
\]

Since $\tanh(y)$ is concave, $s_2$ is convex \todo{proof} on $0 \le y \le Z$, we can approximate it from below using its first derivative at $Z$:

\[
s_3(x + \iu y) = s_2'(Z)(y - Z) = \frac{{\left(W^{2} - 1\right)}^{2}}{4 \, {\left(W^{2} + 1\right)} W} (Z - y)
\]
% \[
% h'(Z) = \frac{{\left(W^{2} - 1\right)}^{2}}{4 \, {\left(W^{2} + 1\right)} W}
% \]
% h'(W=2) = 9/40; h'(W=3) = 8/15

Since $s_3$ is stricly decreasing, $s(0) = \frac{{\left(W^{2} - 1\right)}^{2}}{4 \, {\left(W^{2} + 1\right)} W} Z$ is strictly positive and $s(Z) = 0$, $s_3(k)$ is non-negative and $\ln s_3(k)$ is well defined on $0 \le y \le Z$.

$s_3(k)$ helps us to deal 


\subsection*{Case 2: $y > Z$}
For $y > Z$, 
\[
s_1(x + \iu y) 
 = \frac{\tanh y - \frac{2 W}{W^2 + 1}}{\tanh y + \frac{2 W}{W^2 + 1}}
 = 1 - 2 \frac{\frac{2 W}{W^2 + 1}}{\tanh y + \frac{2 W}{W^2 + 1}}
\]

% TODO
% https://proofwiki.org/wiki/Inverse_of_Strictly_Increasing_Strictly_Concave_Real_Function_is_Strictly_Convex
Since $s_1$ is concave \todo{proof? just take a look at second derivative?}, and strictly increasing, we can take any $Z_0 > Z$ and approximate the function by a piecewise linear function $s_4$:

\[
s_4(x + \iu y) = 
\begin{cases}
\frac{s_1(Z_0)}{Z_0 - Z} (y - Z) &, Z < y \le Z_0 \\
s_1(Z_0) &, y > Z_0
\end{cases}
\]


\subsection*{Proof of convergence for constant part}
\todo{Might be easier in Cayley space}

\subsection*{Proofs of convergence for logarithmic parts}
\todo{DO NOT FORGET ABOUT shifted circle!!! $k = R \eexp{\iu t} + \iu C(R)$}

\todo{For second, we could probably get rid of $\frac{1}{(k-1)^2}$ via the minimum estimation}

\todo{For first, things are slightly harder. Prove about generic function with zero at $Z$, then first implies second???}
Peak induced by $k=1$ is quite nasty. Ensure it converges, implement a checker. Although, it might not, it probably gets smoothed by $\abs{\det S} = 1$ for the original function....


\todo{Use some inequalities involving logarithms???}
% http://dlmf.nist.gov/4.5

\todo{Unit cicle in Cayley space maps to $R$, and $\ln\abs{\det S}$ on $R$ is zero everywhere}

\end{document}