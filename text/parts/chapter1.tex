\chapter{Предварительные сведения}
\label{chapter1}

\section{Уравнение Шредингера}
\todo{рассеиватель}

\todo{резонаторы}

\section{S-матрица}

\todo{картиночку с полюсами и нулями??}

В общем виде волновые функции, являющиеся решениями уравнения Шредингера слева и справа от резонатора имеют вид:
\[
\psi_L(x) = A \eexp{\iu k x} + B \eexp{-\iu k x}
\]
\[
\psi_R(x) = C \eexp{\iu k x} + D \eexp{-\iu k x}
\]

S-матрица (матрица рассеяния) выражает зависимость исходящей волны от входящей:
\[
\begin{pmatrix} B \\ C \end{pmatrix} = S \begin{pmatrix} A \\ D \end{pmatrix}
\]

\todo{свойства}

\todo{симметричный/несимметричный случай}

\section{Преобразование Кэли}

Прямое преобразование Кэли (англ. Cayley transform) конформно отображает верхнюю комплексную полуплоскость (англ. upper half-plane) $\bbH = \{ x + \iu y \mid y > 0, x, y \in \bbR \}$ в комплексный единичный круг (англ. unit disk) $\bbD = \{ z \mid \abs{z} < 1 \}$:
\begin{equation}\label{eq:cayley}
W(z) = \frac{z - \iu}{z + \iu}
\end{equation}
, обратное преобразование Кэли аналогично отображает $\bbD$ в $\bbH$:
\begin{equation}\label{eq:cayley_inverse}
w(\zeta) = \iu \frac{1 + \zeta}{1 - \zeta}
\end{equation}

Важным свойством преобразование Кэли является инъективное отображение $\bbR$ в единичную окружность $\bbT = \partial \bbD =  \{z \mid \abs{z} = 1 \}$

\todo{перенести то что дальше куда-нибудь?}
Так как преобразование Кэли является трансформацией Мебиуса, оно сохраняет окружности. В частности, окружность с радиусом $r$ в нуле, под действием обратного преобразование Кэли перейдет в окружность с центром $C(r)$ и радиусом $R(r)$, где:

\begin{equation}
\begin{aligned}
   C(r) &= \Im \frac{w(r) + w(-r)}{2}
\\ R(r) &= \Im \frac{w(r) - w(-r)}{2}
\end{aligned}
\end{equation}

Легко заметить, что при стремлении $r$ к $1$, $R(r)$ стремится к бесконечности, а $C(r)$ стремится к $R(r)$, что естественно так как в пределе комплексная единичная окнужность отображается в вещественную ось.

\todo{написать про нотацию $z$ vs $\zeta$}

\todo{нотация, может сделать секцию?}

\todo{использовать шаблоны определений, теоремы и т.п.}

\section{Внутренние функции и произведение Бляшке}
Для скалярных внутренних функций $\phi$ определенных на комплексном единичном диске $\bbD$, существует критерий отстутствия сингулярного множителя \cite[стр. 99]{nikolskii}:
\[
\lim\limits_{r = 1} \int\limits_{\bbT} \log \abs{\phi(r \zeta)} d m(\zeta) = 0
\]

Так как $\det S$ является скалярной внутренней функцией, мы можем воспользоваться данным критерием для исследования полноты системы резонантных состояний.

\todo{мутно}

\todo{Неравенство Коши-Шварца??}

\section{Фиксирование нотации}
\subsection{Различные обозначения}

\begin{ilist}
# Жирным обозначаются вектора: к примеру, $\vb{r} \in \bbR^n$, без выделения жирным — их длины: $r = |\vb{r}|$;
# Сопряжение комплексных чисел обозначается как $\cconj{c}$;
# Сопряжение операторов обозначается как $A^*$.
\end{ilist}

\subsection{Скалярное произведение}

В данной работе используется \textbf{физическая} нотация. В частности, это означает, что скалярное произведение в $L^2(E)$ определено как $\ip{f}{g} = \int\limits_E \cconj{f(\vb{x})} g(\vb{x}) \dd \vb{x}$.

\section{Терминология теории рассеяния}
Состояния рассеяния (англ. scattering states) — решения уравнения Шредингера, соответствующие непрерывному спектру, и не лежащие в $L^2$. Также частно называется каналом рассеяния (англ. scattering channel).

Мода (англ. mode) — связанная часть состояния рассеяния. О них обычно говорят в контексте волноводов, имеющих двумерную или трехмерную геометрию, и допускающих только связанные состояния в поперечном направлении. К примеру, в волноводе с конфигурацией $\Omega = [-\infty, \infty] \times [0, H]$ допустимыми поперечными модами являются $\psi_n(y) = \sqrt{\frac{2}{H}} \sin(\frac{\pi n}{H} y)$, $n \in \bbN^+$. Если поперечной частью волны является мода $n$, говорят, что «волна распространяется на $n$-й моде».

Открытый канал (англ. open channel) — канал рассеяния, на котором волна потенциально может распространяться при данной энергии $E$. Закрытый канал (англ. closed channel) — канал рассеяния, на котором волна не может распространяться при данной энергии $E$. Для вышеупомянутого примера двумерного волновода при любой энергии $E$ всегда будут открыты каналы $\{n \mid n \in \bbN^+, \left(\frac{\pi n}{H}\right)^2 < E \}$, которых, очевидно, конечное число.