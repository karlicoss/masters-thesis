\chapter{Обзор предметной области}
\label{chapter1}

\todo{REWRITE!!!}

\section{Аксиоматизация квантовой механики}

\section{Уравнение Шредингера}


\section{Ток вероятности}

\section{Коэффициент прохождения}

\section{Фиксирование нотации}
\subsection{Различные обозначения}

\begin{ilist}
# Жирным обозначаются вектора: к примеру, $\vb{r} \in \bbR^n$, без выделения жирным — их длины: $r = |\vb{r}|$;
# Сопряжение комплексных чисел обозначается как $\cconj{c}$;
# Сопряжение операторов обозначается как $A^*$.
\end{ilist}

\subsection{Атомная система единиц}
% https://en.wikipedia.org/wiki/Atomic_units
% http://www.phys.ubbcluj.ro/~vchis/cursuri/cspm/course2.pdf
В данной работе все расчеты ведутся в атомной системе единиц Хартри (англ. Hartree atomic units, далее АСЕ). В ней нормализуются следующие константы:

\begin{table}[h!]
\begin{tabular}{|l|l|l|}
\hline
Величина & Значение в АСЕ & Значение в СИ \\\hline
Действие & 1 приведенная постоянная Планка & $\approx$ \num{1.05e-34} Дж$\cdot$с \\\hline
Масса & 1 масса электрона &  $\approx$ \num{9.1e-31} кг \\\hline
Заряд & 1 заряд электрона & $\approx$ \num{1.6e-19} Кл \\\hline
\end{tabular}
\end{table}

Производными единицами, необходимыми нам, будут:

\begin{table}[h!]
\begin{tabular}{|l|l|l|}
\hline
Величина & Значение в АСЕ & Значение в СИ \\\hline
Длина & 1 боровский радиус & $\approx$ \num{5.29e-11} м \\\hline
Энергия & 1 хартри &  $\approx$ \num{4.3e-18} Дж \\\hline
\end{tabular}
\end{table}
\textbf{Далее, если не будет оговорено иное, все квантомеханические уравнения и вычисления будут приведены в АСЕ.}

\subsection{Скалярное произведение}

В данной работе используется \textbf{физическая} нотация. В частности, это означает, что скалярное произведение в $L^2(E)$ определено как $\ip{f}{g} = \int\limits_E \cconj{f(\vb{x})} g(\vb{x}) \dd \vb{x}$.

\section{Терминология теории рассеяния}
Состояния рассеяния (англ. scattering states) — решения уравнения Шредингера, соответствующие непрерывному спектру, и не лежащие в $L^2$. Также частно называется каналом рассеяния (англ. scattering channel).

Мода (англ. mode) — связанная часть состояния рассеяния. О них обычно говорят в контексте волноводов, имеющих двумерную или трехмерную геометрию, и допускающих только связанные состояния в поперечном направлении. К примеру, в волноводе с конфигурацией $\Omega = [-\infty, \infty] \times [0, H]$ допустимыми поперечными модами являются $\psi_n(y) = \sqrt{\frac{2}{H}} \sin(\frac{\pi n}{H} y)$, $n \in \bbN^+$. Если поперечной частью волны является мода $n$, говорят, что «волна распространяется на $n$-й моде».

Открытый канал (англ. open channel) — канал рассеяния, на котором волна потенциально может распространяться при данной энергии $E$. Закрытый канал (англ. closed channel) — канал рассеяния, на котором волна не может распространяться при данной энергии $E$. Для вышеупомянутого примера двумерного волновода при любой энергии $E$ всегда будут открыты каналы $\{n \mid n \in \bbN^+, \left(\frac{\pi n}{H}\right)^2 < E \}$, которых, очевидно, конечное число.