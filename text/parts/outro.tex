\startconclusionpage

В работе исследовано поведение системы резонансных состояний различных квантовых графов, в частности, аналитически исследован вопрос полноты. Резонансные состояния таких систем сложно получить в явном виде, и соответственно, доказать полноту системы напрямую. Было показано качественно различное поведение, казалось бы, похожих моделей и показано что полнота системы может быть потеряна в ходе непрерывной трансформации квантового графа и его граничных условий. Это является весьма важным свойством, которое означает что исследование полноты системы резонансных состояний сложного квантового графа не всегда может быть легко сведено к исследованию полноты для более простого квантового графа, полученного, например, стягиванием ребер исходного. \todo{что-нибудь про инвариант индукции?}

Данная работа отличается новизной, так как ранее подобное исследование было проведено только для двух простейших моделей резонаторов \todo{ссылки} 

Результаты данной работы являются частью более глобальной задачи о \todo{что-то тут Попов про физиков рассказывал?}.