\startconclusionpage

\todo{REWRITE}

% В работе предложен метод получения приближенных спектральных и проводящих характеристик двумерного волновода сложной структуры, основанный на самосопряженных расширениях симметрических операторов и выходе в понтрягинское пространство функций. Для моделей подобных конфигураций не существует аналитических решений, способы численного решения автору также неизвестны.

% Данная работа отличается новизной, так как ранее подобные вычисления были проделаны только для волноводов с граничным условием Неймана \cite{popov1993zero}, которое, во-первых, не является физически обоснованным, а во-вторых, значительно проще для вычислений в рамках теории расширений симметрических операторов в том смысле, что расчеты могут быть проделаны в пространстве $\mcL^2$.

% На основе полученных результатов сделан вывод, что данная модель подходит в качестве возможной реализации наноразмерных транзисторов, непосредственно использующих квантовые эффекты. Построена зависимость коэффициента прохождения от энергии входящей волны и обнаружены резонансы. Построена зависимость коэффициента прохождения от геометрии резонатора, на которой также обнаруживается резонанс, что позволяет использовать данную нелинейность в коэффициенте прохождения для реализации транзистора.

% Результаты данной работы также могут быть использованы в расчетах расширений для более сложных квантомеханических операторов, в частности, оператора Дирака, учитывающего релятивистские эффекты, и более сложных моделей, например, трехмерных волноводов.

% \todo{Список литераторы: сделать P. вместо С.}