\startconclusionpage

В работе исследовано поведение системы резонансных состояний различных квантовых графов, в частности, аналитически исследован вопрос полноты. Резонансные состояния таких систем сложно получить в явном виде, и соответственно, доказать полноту системы напрямую. Данная работа отличается новизной, так как ранее подобное исследование было проведено только для двух простейших моделей резонаторов \cite{spectralpavlov16}.

В \autoref{sec:bundle} была доказана полнота резонансных состояний для квантового графа сложной формы. Выработанный подход к доказательству может быть обобщен и использован для доказательства полноты для других граничных условий и квантовых графов.

В \autoref{sec:ring} было показано качественно различное поведение, казалось бы, похожих моделей и показано что полнота системы может быть потеряна в ходе непрерывной трансформации квантового графа и его граничных условий. Это является весьма важным свойством, которое означает что исследование полноты системы резонансных состояний сложного квантового графа не всегда может быть легко сведено к исследованию полноты для более простого квантового графа, полученного, например, стягиванием ребер исходного, и возможно, для исследования полноты произвольного квантового графа необходим некий сложный шаг индукции, если не другой подход. Также заметим, что в произвольном квантовом графе ребра могут соединять либо две разные вершины, либо быть петлей, что и является резонатором типа «кольцо», исследованным в данной работе. 

Некоторые из результатов работы были опубликованы в \cite{sbornik} и приняты для презентации презентацию на конференции \cite{diffdays}.

Проведенное исследование является важной подзадачей в более общей задаче нахождения пространства, в котором резонансные состояния квантового графа будут образовывать полную систему.

Естественным продолжением данной работы может являться исследование полноты системы резонансных состояний при других граничных условиях в точке соединения резонатора и волновода.