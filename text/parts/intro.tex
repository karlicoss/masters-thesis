\startprefacepage

Исследования резонансов и резонансных состояний различных физических систем проводились давно, начиная с классической работы лорда Рэлея \cite{rayleigh1916theory}, и по наши дни. Однако, строгое математическое обоснование и формализм этим являниям были даны во второй половине XX века. Теперь известно, что резонансы являются собственными числами, а резонансные состояние — собственными функциями некого диссипативного оператора \cite{lax1990scattering, adamjan1965class}.

Одним из интересных как с математической, так и физической точек зрения вопросов является вопрос о полноте резонансных состояний. Рассмотрим некий конечный резонатор $\Gamma$ с граничным условием Дирихле или Неймана. Как хорошо известно из теории операторов и рассеяния \cite{reed1980methods}, оператор Шредингера в такой области будет обладать чисто дискретным спектром и полной системой собственных функций в $\mcL^2(\Gamma)$. Теперь применим к резонатору возмущение, соединив его через небольшое отверстие с волноводом. После этого, у системы будет непрерывный спектр, а собственные значения исходной системы превращаются в резонансы и «смещаются» в комплексную плоскость. Резонансные состояния формально удовлетворяют уравнению Шредингера и граничным условиям, однако не квадратично интерируемы, так как их волновые функции не уходят в ноль на бесконечности, то есть не являются собственными функциями оператора Шредингера на новом домене. Однако, при сужении этих резонасных состояний на конечный домен $\Omega$, они вновь становятся интегрируемыми и принадлежат простанству $\mcL_2(\Omega)$. Интересным является вопрос о полноте этих суженных состояний на домене $\Omega$, и соответственно, вопрос о поиске такого поддомена квантовой системы, на котором его резонансные состояния будут полны. Существует гипотеза о том, что таким поддоменом является выпуклая оболочка рассеивателя. \todo{сослаться на работы которые уже были?}

% Квантовый граф — широко используемая модель наносистемы \cite{kuchment2002graph, lobanov2013genetic, 3, 4}. Если граф $\Gamma$ состоит только из конечного числа ребер конечной длины, его гамильтониан имеет чисто дискретный спектр, и его собственные функции образуют полную систему в $\mcL_2(\Gamma)$. Если же граф $\Gamma$ представляет из себя резонатор $\Omega$ с полубесконечными ребрами, в спектре будет присутствовать непрерывная часть и резонансы, индуцированные собственными числами гамильтониана резонатора $\mcH_\Omega$. Резонансные состояния оператора Шредингера не принадлежат пространству $\mcL_2(\Gamma)$, однако, при сужении их на конечный домен $\Omega$, становятся квадратично интегрируемыми и лежат в пространстве $\mcL_2(\Omega)$. Для многих приложений интересно знать, формируют ли полную систему резонанстные состояния графа $\Gamma$ в пространстве $\mcL_2(\Omega)$ резонатора.

В статье \cite{popov_exner70} была показана полнота резонансных функций на графе с резонатором, состоящим из отрезка, связанным с волноводом $\delta$-образным барьером. Естественным предположением является что произвольный резонатор состоящий из конечного числа ребер конечной длины должен обладать похожими свойствами, однако, мы покажем, что это не так. 

Целью данной работы является исследование нескольких моделей квантовых графов и анализ их резонансных состояний на предмет их полноты на области резонатора.