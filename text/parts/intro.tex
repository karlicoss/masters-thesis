\startprefacepage

\todo{REWRITE}

% В микроэлектронике для изготовления интегральных схем в основном используются полевые транзисторы. Полевой транзистор — прибор, в простейшем случае состоящий из трех контактов:
% \begin{easylist}[itemize]
% # исток — контакт, на который подаются носители заряда;
% # сток — контакт, с которого уходят носители заряда;
% # затвор — контакт, напряжением на котором можно регулировать ток, идущий от истока к стоку.
% \end{easylist}
% Фактически, транзистор является «управляемым сопротивлением», то есть прибором, проводимость которого можно контролировать напряжением на затворе, что и позволяет использовать его для реализации логических элементов.

% К сожалению, исследовать аналитически квантомеханические системы очень сложно, к примеру, уже в простейшей модели одномерной прямоугольной квантовой ямы (англ. 1D finite potential well) для расчета собственных энергий и функций, необходимых для расчета коэффициента проводимости, нужно решать трансцендентные уравнения. Естественно, в двумерном и трехмерном случаях, уравнения решать тем более сложнее.

% В рамках решения этой проблемы разработан подход, позволяющий изменить исходную модель, получив ее аппроксимацию, которая допускает аналитическое решение. В работах \cite{popov1992extension, popov1992resonator, popov1993zero} этот подход был применен для граничного условия Неймана. Однако, физически обоснованным граничным условием является граничное условие Дирихле, которое еще не было исследовано в рамках этого подхода.

% Целью данной работы является исследование квантового волновода с граничным условием Дирихле, аналитического решения уравнения Шредингера для этого волновода методом аппроксимации моделью нулевого радиуса и расчет его спектральных характеристик.