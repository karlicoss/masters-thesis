\startprefacepage

\todo{REWRITE}


% Квантовый граф — широко используемая модель наносистемы \cite{kuchment2002graph, lobanov2013genetic, 3, 4}. Если граф $\Gamma$ состоит только из конечного числа ребер конечной длины, его гамильтониан имеет чисто дискретный спектр, и его собственные функции образуют полную систему в $\mcL_2(\Gamma)$. Если же граф $\Gamma$ представляет из себя резонатор $\Omega$ с полубесконечными ребрами, в спектре будет присутствовать непрерывная часть и резонансы, индуцированные собственными числами гамильтониана резонатора $\mcH_\Omega$. Резонансные состояния оператора Шредингера не принадлежат пространству $\mcL_2(\Gamma)$, однако, при сужении их на конечный домен $\Omega$, становятся квадратично интегрируемыми и лежат в пространстве $\mcL_2(\Omega)$. Для многих приложений интересно знать, формируют ли полную систему резонанстные состояния графа $\Gamma$ в пространстве $\mcL_2(\Omega)$ резонатора.

% \section{Цель работы и постановка задачи}
В статье \cite{popov_exner70} была показана полнота резонансных функций на графе с резонатором, состоящим из отрезка, связанным с волноводом $\delta$-образным барьером. Естественным предположением является что произвольный резонатор состоящий из конечного числа ребер конечной длины должен обладать похожими свойствами, однако, это не так. В этой работе исследуется резонатор, состоящий из кольца и показывается неполнота резонансных состояний на этом кольце.


\begin{itemize}
\item Задача полноты важна как в теории, так и на практике.
% Задача полноты важна и с теоретической точки зрения (теория функций, дифференциальные уравнения), так и с практической (возможность разложить состояния в ряд Фурье, аппроксимации и т.п.)

\item Рассматриваем резонатор и оператор Шредингера с граничным условием Дирихле (чисто дискретный спектр).
% Рассматривается некий резонатор, оператор Шредингера с граничным условием Дирихле в нем имеет чисто дискретный спектр, набор собственный функций полон в $\mcL^2(\Omega)$.

\item Применяем возмущение, соединяя резонатор с волноводом (дискретный спектр смещается в резонансы)

\item Интересным вопросом является нахождение домена, на котором резонансные состояния полны. \textbf{Гипотеза: на выпуклой оболочке рассеивателя}.
\end{itemize}

\todo{ссылка на Sz. Nagy??}
% B. Sz.-Nagy, C. Foias, H. Bercoviuci, L. Kerchy, Harmonic Analysis of Operators on Hilbert Space,
% 2nd edition, Berlin: Springer (2010)

\todo{Куда существующие результаты?}

\todo{Картинки с резонатором из презентации}

% 
% Далее, рассматриваем возмущение этой задачи, соединяя резонатор с волноводом через небольшое отверстие. После этого дискретный спектр пропадает, а собственные значения «смещаются» в комплексную плоскость и становятся резонансами. Резонансные состояния формально удовлетворяют уравнению Шредингера и граничным условиям, однако не принадлежат L^2(Gamma). С другой стороны, при ограничении их на конечный домен Omega, они лежат в L^2(Omega). 

% \todo{Картинка с резонатором}

% \todo{Картинка с резонатором и проводами}

% \todo{Картинка с резонансами, верхняя комплексная плоскость}

% В микроэлектронике для изготовления интегральных схем в основном используются полевые транзисторы. Полевой транзистор — прибор, в простейшем случае состоящий из трех контактов:
% \begin{easylist}[itemize]
% # исток — контакт, на который подаются носители заряда;
% # сток — контакт, с которого уходят носители заряда;
% # затвор — контакт, напряжением на котором можно регулировать ток, идущий от истока к стоку.
% \end{easylist}
% Фактически, транзистор является «управляемым сопротивлением», то есть прибором, проводимость которого можно контролировать напряжением на затворе, что и позволяет использовать его для реализации логических элементов.

% К сожалению, исследовать аналитически квантомеханические системы очень сложно, к примеру, уже в простейшей модели одномерной прямоугольной квантовой ямы (англ. 1D finite potential well) для расчета собственных энергий и функций, необходимых для расчета коэффициента проводимости, нужно решать трансцендентные уравнения. Естественно, в двумерном и трехмерном случаях, уравнения решать тем более сложнее.

% В рамках решения этой проблемы разработан подход, позволяющий изменить исходную модель, получив ее аппроксимацию, которая допускает аналитическое решение. В работах \cite{popov1992extension, popov1992resonator, popov1993zero} этот подход был применен для граничного условия Неймана. Однако, физически обоснованным граничным условием является граничное условие Дирихле, которое еще не было исследовано в рамках этого подхода.

% Целью данной работы является исследование квантового волновода с граничным условием Дирихле, аналитического решения уравнения Шредингера для этого волновода методом аппроксимации моделью нулевого радиуса и расчет его спектральных характеристик.