\startprefacepage

Исследования резонансов и резонансных состояний различных физических систем проводились давно, начиная с классической работы лорда Рэлея \cite{rayleigh1916theory}, и по наши дни. Однако, строгое математическое обоснование и формализм этим явлениям были даны во второй половине XX века. Теперь известно, что резонансы являются собственными числами, а резонансные состояние — собственными функциями некого диссипативного оператора \cite{lax1990scattering, adamjan1965class}.

Одним из интересных как с математической, так и физической точек зрения вопросов является вопрос о полноте резонансных состояний. Рассмотрим некий конечный резонатор $\Gamma$ с граничным условием Дирихле или Неймана. Как хорошо известно из теории операторов и рассеяния \cite{reed1980methods}, оператор Шредингера в такой области будет обладать чисто дискретным спектром и полной системой собственных функций в $\mcL^2(\Gamma)$. Теперь применим к резонатору возмущение, соединив его через небольшое отверстие с волноводом. После этого, у системы будет непрерывный спектр, а собственные значения исходной системы превращаются в резонансы и «смещаются» в комплексную плоскость. Резонансные состояния формально удовлетворяют уравнению Шредингера и граничным условиям, однако не квадратично интегрируемы, так как их волновые функции не уходят в ноль на бесконечности, то есть не являются собственными функциями оператора Шредингера на новом домене. Однако, при сужении этих резонансных состояний на конечный домен $\Omega$, они вновь становятся интегрируемыми и принадлежат пространству $\mcL_2(\Omega)$. Интересным является вопрос о полноте этих суженных состояний на домене $\Omega$, и соответственно, вопрос о поиске такого поддомена квантовой системы, на котором его резонансные состояния будут полны. Существует гипотеза о том, что таким поддоменом является выпуклая оболочка рассеивателя.

\begin{figure}
\renewcommand\thefigure{А}
\centering
\begin{tikzpicture}[scale=1.1]
\newcommand{\Wglen}{10.0}; % waveguide length

\coordinate (LLL) at (-\Wglen - 1, 0);
\coordinate (RRR) at ( \Wglen + 1, 0);
\coordinate (LL)  at (-\Wglen, 0);
\coordinate (RR)  at ( \Wglen, 0);
% change these to take resonator length into account
\coordinate (L)   at (0, 0);
\coordinate (R)   at (0, 0);
%

\coordinate (U) at (0, 3); % upper point of resonator

% waveguide
\draw[ultra thick, dotted] (LLL) -- (LL);
\draw[ultra thick] (LL)--(L);
\draw[ultra thick] (R)--(RR);
\draw[ultra thick, dotted] (RR) -- (RRR);
%

\draw[<-] (-\Wglen + 1, 2) -- (-\Wglen + 5, 2) node [midway, above] {$R e^{-\iu k x}$};
\draw[->] (-\Wglen + 1, 0.5) -- (-\Wglen + 5, 0.5) node [midway, above] {$e^{\iu k x}$};
\draw[->] (\Wglen - 5, 0.5) -- (\Wglen - 1, 0.5) node [midway, above] {$T e^{\iu k x}$};

\draw[ultra thick] node [below] {$V$} (0, 0) -- node [left] {$\Omega$} (U);
\end{tikzpicture}
\caption{Одномерный резонатор-отрезок с условием Дирихле на краю резонатора.} \label{fig:res_segment}
\end{figure}

\begin{figure}
\renewcommand\thefigure{Б}
\centering
\begin{tikzpicture}[scale=1.1]
\newcommand{\Wglen}{6.0}; % waveguide length
\newcommand{\Warrlen}{2.0}; % wave arrow length
\newcommand{\Reslen}{2.0}; % resonator length

\coordinate (LLL) at (-\Wglen - 1, 0);
\coordinate (RRR) at ( \Wglen + 1, 0);
\coordinate (LL)  at (-\Wglen, 0);
\coordinate (RR)  at ( \Wglen, 0);
\coordinate (L)   at (-\Reslen, 0);
\coordinate (R)   at ( \Reslen, 0);
%

\coordinate (U) at (0, 3); % upper point of resonator

% waveguide
\draw[ultra thick, dotted] (LLL) -- (LL);
\draw[ultra thick] (LL)--(L);
\draw[ultra thick] (R)--(RR);
\draw[ultra thick, dotted] (RR) -- (RRR);
%

\draw[<-, thick] (-\Wglen, 1.5) -- (-\Wglen + \Warrlen, 1.5) node [midway, above] {\large $R e^{-\iu k x}$};
\draw[->, thick] (-\Wglen, 0.8) -- (-\Wglen + \Warrlen, 0.8) node [midway, above] {\large $e^{\iu k x}$};
\draw[->, thick] (\Wglen - \Warrlen, 0.5) -- (\Wglen, 0.5)   node [midway, above] {\large $T e^{\iu k x}$};

\draw (LL) node [below] {\large $\Omega_L$};
\draw (RR) node [below] {\large $\Omega_R$};

\draw[ultra thick] (0, 2) node [above] {\Large $\Omega$};

\draw (-2,0) arc (180:360:2cm and 0.5cm);
\draw[dashed] (-2,0) arc (180:0:2cm and 0.5cm);
\draw (0,2) arc (90:270:0.5cm and 2cm);
\draw[dashed] (0,2) arc (90:-90:0.5cm and 2cm);
\draw (0,0) circle (2cm);
\shade[ball color=blue!10!white,opacity=0.20] (0,0) circle (2cm);
\end{tikzpicture}
\caption{Трехмерный резонатор (квантовая точка).} \label{fig:res_3d}
\end{figure}

В работе \cite{spectralpavlov16} была показана полнота резонансных функций на графе с резонатором, состоящим из отрезка (\autoref{fig:res_segment}), и резонатором в виде квантовой точки (\autoref{fig:res_3d}). Естественным предположением является что произвольный резонатор, состоящий из конечного числа ребер конечной длины должен обладать похожими свойствами, однако, мы покажем, что это не так.

Целью данной работы является исследование нескольких моделей квантовых графов и анализ их резонансных состояний на предмет их полноты на области резонатора.
