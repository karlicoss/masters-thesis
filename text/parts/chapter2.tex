\chapter{Описание реализованного подхода}
\label{chapter2}

\section{Модель резонатора типа «пучок»} % MINOR TODO пучок? =/
\todo{описание}

\todo{figure}

\todo{Определитель $S$-матрицы в замкнутой форме}


\section{Модель резонатора типа «кольцо»}

\todo{описание}

\todo{figure}

Определитель $S$-матрицы в замкнутой форме:

\[
\det S = 
\frac
{\cos\left(k\right) + {\left(\frac{a}{2 k} + i\right)} \sin\left(k\right) - 1}
{\cos\left(k\right) + {\left(\frac{a}{2 k} - i\right)} \sin\left(k\right) - 1}
\]

\subsection{Исследование случая $a = 0$}
\todo{!!!}

\subsection{Исследование случая $a \ne 0$}
\todo{!!!}

\todo{Объяснить результат}

\todo{другие виды резонаторов?}

\section{Собственные состояния, энергии и функции Грина областей задачи}
Для дальнейших расчетов понадобятся решения следующих задач:

\subsection{Одномерная яма с бесконечными стенками}


\subsection{Одномерная свободная частица}


\subsection{Двумерная яма с бесконечными стенками}

\subsection{Бесконечная квазиодномерная полоса с бесконечными стенками}


\section{Асимптотика производной функции Грина в окрестности отверстия}


\section{Построение модели с отверстием нулевой ширины}
 в котором будет корректно определено скалярное произведение на них.

\subsection{Расчет понтрягинского пространства для резонатора}
\subsubsection{Расчет дефектного элемента}

\subsubsection{Построение предпонтрягинского пространства}
