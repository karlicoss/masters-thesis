\subsection{Модель исследования}
% TODO http://tex.stackexchange.com/a/155317/5966 ?
\begin{figure}[!htb] % TODO maybe just ref instead?
\begin{tikzpicture}[scale=0.8] % TODO SCALE!!!
\newcommand{\Wglen}{6.0}; % waveguide length
\newcommand{\Warrlen}{2.0}; % wave arrow length
\newcommand{\Reslen}{0.0}; % resonator length

\coordinate (LLL) at (-\Wglen - 1, 0);
\coordinate (RRR) at ( \Wglen + 1, 0);
\coordinate (LL)  at (-\Wglen, 0);
\coordinate (RR)  at ( \Wglen, 0);
\coordinate (L)   at (-\Reslen, 0);
\coordinate (R)   at ( \Reslen, 0);
%

\coordinate (U) at (0, 3); % upper point of resonator

% waveguide
\draw[ultra thick, dotted] (LLL) -- (LL);
\draw[ultra thick] (LL)--(L);
\draw[ultra thick] (R)--(RR);
\draw[ultra thick, dotted] (RR) -- (RRR);
%

\draw[<-, thick] (-\Wglen, 1.5) -- (-\Wglen + \Warrlen, 1.5) node [midway, above] {\large $R e^{-\iu k x}$};
\draw[->, thick] (-\Wglen, 0.8) -- (-\Wglen + \Warrlen, 0.8) node [midway, above] {\large $e^{\iu k x}$};
\draw[->, thick] (\Wglen - \Warrlen, 0.5) -- (\Wglen, 0.5)   node [midway, above] {\large $T e^{\iu k x}$};

\draw (LL) node [below] {\large $\Omega_L$};
\draw (RR) node [below] {\large $\Omega_R$};

\draw[ultra thick] (0, 2) node [above] {\Large $\Omega$};

\draw[ultra thick] node [below] {\large $V$} (0, 2) circle[radius=2];
\end{tikzpicture}
\caption{Квантовый граф $\Gamma$, состоящий из полубесконечных ребер $\Omega_L, \Omega_R$ и рассеивателя $\Omega$, представляющего из себя окружность длиной $1$.}
\end{figure}

Мы рассматриваем случай рассеяния волны c волновым вектором $k$, приходящей слева направо. Таким образом, волновые функции на различных частях графа принимают следующий вид:

\begin{equation}\label{eq:ring_system}
\begin{aligned}
\psi_L(x) &= \eexp{\iu k x} + R \eexp{-\iu k x} \\
\psi_R(x) &= T \eexp{\iu k x}\\
\psi_\Omega(x) &= P \sin(k x) + Q \cos(k x)
\end{aligned}
\end{equation}
, где $R$ и $T$ — коэффициенты отражения и прохождения волны. Так как граф симметричен, его матрица рассеяния принимает вид
$S(k) = \begin{pmatrix} R(k) & T(k) \\ T(k) & R(k) \end{pmatrix}$.

В вершине $V$ мы ставим $\delta$-образную потенциальную яму высотой $a$, которая порождает следущие граничные условия:

\begin{equation}\label{eq:ring_bc}
\begin{aligned}
\psi_L(0) = \psi_R(0) = \psi_\Omega(0) = \psi_\Omega(1) \\ 
-\psi'_L(0) + \psi'_\Omega(0) - \psi'_\Omega(1) + \psi'_R(0) = a \psi_L(0)
\end{aligned}
\end{equation}


\subsection{Вычисление S-матрицы}
Определим коэффициенты прохождения и отражения, подставив функции (\ref{eq:ring_system}) в (\ref{eq:ring_bc}) и решив систему линейных уравнений:
\begin{align*}
& 1 + R &= T \\
& 1 + R &= Q \\
& Q \cos k + P \sin k &= T \\
& -P k \cos k + Q k \sin k + P k + \iu R k + \iu T k - \iu k &= T a
\end{align*}

Решив систему, получаем:

\begin{align*}
R(k) = -\frac{2 \, k \cos\left(k\right) + a \sin\left(k\right) - 2 \, k}{2 \, k \cos\left(k\right) + {\left(a - 2 i \, k\right)} \sin\left(k\right) - 2 \, k} \\
T(k) = -\frac{2 i \, k \sin\left(k\right)}{2 \, k \cos\left(k\right) + {\left(a - 2 i \, k\right)} \sin\left(k\right) - 2 \, k}
\end{align*}
, подставляя полученные значения коэффициента прохождения и отражения в S-матрицу, получаем определитель S-матрицы в замкнутой форме:

\begin{equation}\label{eq:ring_detS}
\det S = 
\frac
{\cos\left(k\right) + {\left(\frac{a}{2 k} + i\right)} \sin\left(k\right) - 1}
{\cos\left(k\right) + {\left(\frac{a}{2 k} - i\right)} \sin\left(k\right) - 1}
\end{equation}


\subsection{Исследование полноты при $a=0$}
Рассмотрим случай $a=0$:
\[
\det S
= \frac
{\cos\left(k\right) + \iu \sin\left(k\right) - 1}
{\cos\left(k\right) - \iu \sin\left(k\right) - 1}
= \frac{\eexp{\iu k} - 1}{\eexp{-\iu k} - 1}
= -\eexp{i k}
\]

Из выражения выше легко заметить, что подынтегриальное выражение в (\todo{ссылку}) сводится к $\ln \abs{\det S} = \ln \eexp{- \Im k} = -\Im k$. Вычислим интеграл в пространстве единичного диска, для этого применим обратное преобразование Кэли $w(\zeta) = \iu \frac{1 + \zeta}{1 - \zeta}$, к подынтегральной функции: $\Im k \to \Im \left( \iu \frac{1 + \zeta}{1 - \zeta} \right) $.

\[
  \lim\limits_{R = 1} \int\limits_{\abs{\zeta} = R} \ln \abs{\det S(\zeta)} d \zeta
= \lim\limits_{R = 1} \int\limits_{\abs{\zeta} = R} \Im \left( \iu \frac{1 + \zeta}{1 - \zeta} \right)  d\zeta = \dots
\]
, параметризуем контур интегрирования полярными координатами: $\zeta \to R \eexp{\iu \phi}, d\zeta \to R \iu \eexp{\iu \phi}$:
\[
\dots = \lim\limits_{R = 1} \int\limits_{\abs{\zeta} = R} \Im \left( \iu \frac{1 + R \eexp{\iu \phi}}{1 - R \eexp{\iu \phi}} \right) R \iu \eexp{\iu \phi} d\phi
\]

Комплексный интеграл складывается из сумм интегралов действительной и мнимной части подынтегрального выражения. Рассчитаем мнимую часть:

\begin{align*}
   \Im \left(  \Im \left( \iu \frac{1 + R \eexp{\iu \phi}}{1 - R \eexp{\iu \phi}} \right) R \iu \eexp{\iu \phi} \right)
\\ &= R \Re \left(  \Re \left( \frac{1 + R \eexp{\iu \phi}}{1 - R \eexp{\iu \phi}} \right) \eexp{\iu \phi} \right)
\\ &= R \Re \left( \frac{1 + R \eexp{\iu \phi}}{1 - R \eexp{\iu \phi}} \right) \Re \left(   \eexp{\iu \phi} \right)
\\ &= R \Re \left( \frac{(1 + R \eexp{\iu \phi}) (1 - R \eexp{-\iu \phi}) }{(1 - R \eexp{\iu \phi}) (1 - R \eexp{-\iu \phi})} \right) \cos \phi
\\ &= R \Re \left( \frac{1 - R^2 + 2 \iu R \sin \phi}{1 + R^2 - 2 R \cos \phi} \right) \cos \phi
\\ &= R \frac{1 - R^2}{1 + R^2 - 2 R \cos \phi} \cos \phi 
% \\ &= \frac{1 - R^2}{2} \frac{\cos \phi}{\frac{1 + R^2}{2R} - \cos \phi} 
\end{align*}

\todo{доинтегрировать}

Интегрируя, получаем $2 \pi R^2$, что значит, что предел мнимой части при $R \to 1$ равен $2 \pi$, следовательно, по критерию полноты, система резонатных состояний графа $\Gamma$ не является полной на кольце $\Omega$.

\todo{график S-матрицы?}


\subsection{Исследование полноты при $a \ne 0$}
\todo{опровергнуть!!}

\todo{Объяснить результат}

\todo{другие виды резонаторов?}