\chapter{Предварительные сведения}

\section{Различные обозначения}
\begin{itemize}
\item $\bbC$ — комплексная плоскость, $\bbC = \{ x + \iu y \mid x, y \in \bbR \}$ 
\item $\bbH$ — верхняя комплексная полуплоскость (англ. upper half-plane), $\bbH = \{ x + \iu y \mid y > 0, x, y \in \bbR \}$
\item $\bbD$ — комплексный единичный круг (англ. unit disk), $\bbD = \{ z \mid \abs{z} < 1 \}$
\item $\bbT$ — комплексная единичная окружность (англ. unit circle), $\bbT = \partial \bbD =  \{z \mid \abs{z} = 1 \}$
\item $z$ используется в качестве переменной, обозрачающей точку в верхней комплексной полуплоскости $\bbH$
\item $\zeta$ используется в качестве переменной, обозрачающей точку на комплексном единичном диске $\bbD$
\end{itemize}

\section{Уравнение Шредингера}

Уравнение Шредингера описывает временную эволюцию чистых состояний квантовых систем во времени, и выглядит как:
\[
\iu \hbar \pdv{t} \Psi(t) = H \Psi(t)
\]
, $\hbar$ — приведенная постоянная Планка, $H$ — гамильтониан, оператор полной энергии системы. Спектром гамильтониана является множество возможных значений энергии квантовой системы, которые можно получить при ее измерении.

Так как в этой работе нам интересна математическая задача рассеяния, и качественное поведение системы, а не конкретные значения энергий, далее мы будем игнорировать постоянную Планка и примем ее равной $1$; а так система статична, и гамильтониан не зависит от времени, можно перейти к стационарному уравнению Шредингера:
\[
H \Psi = E \Psi
\]
При этом решения уравнения разделяются на два класса: связанные состояния (англ. bound states) — те, которые соответствуют дискретной части спектра и чью собственные функции являются квадратично интегрируемыми, и состояния рассеяния (англ. scattering states), которые являются лишь формальными решениями, и соответствуют непрерывной части спектра. Такие решения не лежат не являются квадратично интегрируемыми, и в них мы будем заинтересованы в данной работе.

\section{S-матрица}\label{sec:smatrix}
% \mtodo{картиночку с полюсами и нулями??}
Рассмотрим локализованный одномерный потенциальный барьер или резонатор. Пусть на барьер слева и справа направлены частицы с волновым вектором $k$ (который будет скаляром в одномерном случае). Слева и справа от резонатора частицы ведут себя как свободные, соответственно, в общем виде их волновые функции имеют следующий вид:

\begin{equation}\label{eq:wlwl}
\begin{aligned}
   \psi_L(x) &= A \eexp{\iu k x} + B \eexp{-\iu k x}
\\ \psi_R(x) &= C \eexp{\iu k x} + D \eexp{-\iu k x}
\end{aligned}
\end{equation}

S-матрица, или  матрица рассеяния (англ. scattering matrix) выражает зависимость исходящего состояния от входящего:
\begin{equation}\label{eq:smatrix}
\begin{pmatrix} B \\ C \end{pmatrix} = S \begin{pmatrix} A \\ D \end{pmatrix}
\end{equation}
, и полностью характеризует рассеивающие свойства потенциального барьера.

$S$-матрица обладает рядом интересных свойств \cite[стр. 75]{perelomov1998quantum}, позволяющих анализировать рассеяние в квантовой системе: к примеру, $S$-матрица как функция комплексного аргумента, аналитична в верхней комплексной полуплоскости, а ее полюса соответствуют связным состояниям или резонансам.

\section{Квантовые графы}
Квантовый граф (англ. quantum graph) — широко используемая модель наносистемы \cite{kuchment2002graph, lobanov2013genetic, brown2010analysis}. Квантовый граф представляет из себя набор вершин и линейных, одномерных ребер, их соединяющих, или уходящих на бесконечность (можно обратиться к \autoref{fig:res_segment} в качестве простейшего примера). При этом на ребрах графа действует дифференциальный оператор, к примеру, оператор Шредингера свободной частицы: $-\dv[2]{\psi}{x}$, а в вершинах графа установлено граничное условие, связывающее волновые фукнкции на смежных ребрах.

% Заметим, что если граф $\Gamma$ состоит только из конечного числа ребер конечной длины, его гамильтониан имеет чисто дискретный спектр, и его собственные функции образуют полную систему в $\mcL_2(\Gamma)$. Если же граф $\Gamma$ представляет из себя резонатор $\Omega$ с полубесконечными ребрами, в спектре будет присутствовать непрерывная часть и резонансы, индуцированные собственными числами гамильтониана резонатора $\mcH_\Omega$. Резонансные состояния оператора Шредингера не принадлежат пространству $\mcL_2(\Gamma)$, однако, при сужении их на конечный домен $\Omega$, становятся квадратично интегрируемыми и лежат в пространстве $\mcL_2(\Omega)$. Для многих приложений интересно знать, формируют ли полную систему резонанстные состояния графа $\Gamma$ в пространстве $\mcL_2(\Omega)$ резонатора.

\section{Преобразование Кэли}

Прямое преобразование Кэли (англ. Cayley transform) конформно отображает $\bbH$ в $\bbD$:
\begin{equation}\label{eq:cayley}
W(z) = \frac{z - \iu}{z + \iu}
\end{equation}
, обратное преобразование Кэли аналогично отображает $\bbD$ в $\bbH$:
\begin{equation}\label{eq:cayley_inverse}
w(\zeta) = \iu \frac{1 + \zeta}{1 - \zeta}
\end{equation}

Важным свойством преобразование Кэли является инъективное отображение $\bbR$ в единичную окружность $\bbT$.

Так как преобразование Кэли является трансформацией Мебиуса, оно сохраняет окружности. В частности, окружность с радиусом $r$ в нуле, под действием обратного преобразование Кэли перейдет в окружность с центром в $C(r)$ и радиусом $R(r)$, где:

\begin{equation}\label{eq:c_and_r}
\begin{aligned}
   C(r) &= \Im \frac{w(r) + w(-r)}{2}
\\ R(r) &= \Im \frac{w(r) - w(-r)}{2}
\end{aligned}
\end{equation}

Легко заметить, что при стремлении $r$ к $1$, $R(r)$ стремится к бесконечности, а $C(r)$ стремится к $R(r)$, что естественно так как в пределе комплексная единичная окнужность отображается в вещественную ось.

\section{Теория Лакса-Филипса}
Мы рассматриваем рассеяние в подходе Лакса-Филипса, кратко опишем его (подробно с ним можно ознакомиться в работах \cite{lax1990scattering,lax1976scattering}).

Расмотрим задачу Коши для волнового уравнения в области $\Gamma$:
\begin{equation}
\begin{aligned}
   u''_{tt}   &= H u
\\ u(x, 0)    &= u_0(x)
\\ u'_t(x, 0) &= u_1(x).
\end{aligned}
\end{equation}

Пусть $\mathcal{E}$ — гильбертово пространство двухкомпонентных функций с конечной энергией $(u_0, u_1)$ на графе $\Gamma$:
\[
\norm{(u_0, u_1)}^2_{\mathcal{E}} = \frac{1}{2} \int\limits_\Gamma (\abs{u'_0}^2 + \abs{u_1}^2) dx
\]
, где пара $(u_0, u_1)$ называется данными Коши (англ. Cauchy data), а оператор $U(t)$, дающий решение этой задаче: $U(t)(u_0, u_1) = (u(x, t), u'_t(x, t))$, порождает унитарную группу $\{ U(t) \mid t \in \bbR \}$. Данная группа содержит два ортогональных подпространства: $D_-$ — входящее подпространство, и $D_+$ — исходящее.

В теории рассеяния установлено, что существуют отображения $T_\pm: \mathcal{R} \to \mcL^2(\bbR, \bbC^2)$, такие что $T_\pm U(t) = \eexp{\iu k t T_\pm}$, которые называются исходящим (соответственно, входящим) спектральным представлением унитарной группы $U(t)$. Обозначим $K = \mathcal{E} \ominus (D_+ \oplus D_-)$, и рассмотрим полугруппу $Z(t) = \mcP_K U(t)|_K, t > 0$, где $\mcP_K$ — оператор проецирования на $K$. Пусть $B$ — генератор $Z(t)$, то есть $Z(t) = \eexp{\iu B t}$, тогда собственные вектора этого оператора и будут резонансными состояниями, а $T_- T_+^{-1}$ называется оператором рассеяния. Оператор рассеяния действует как умножения на матрично-значную функцию $S(k)$, и являющейся матрицей рассеяния (см. \autoref{sec:smatrix}).

\begin{theorem}[{\cite[стр. 95]{nikol2012treatise}}]
Пусть $S$ — внутренняя функция, тогда следущие утверждения эквивалентны:
\begin{itemize}
\item Оператор $Z$ полный.
\item $S$ — произведение Бляшке-Потапова.
\end{itemize}
\end{theorem}

\begin{theorem}[{\cite[стр. 99]{nikol2012treatise}}]
Пусть вспомогательное пространство конечномерно. Тогда, следущие утверждения эквивалентны:
\begin{itemize}
\item $S$ — произведение Бляшке-Потапова.
\item
\begin{equation}\label{eq:blaschke}
\lim\limits_{r = 1} \int\limits_{\bbT} \log \abs{\det S(r \zeta)} d m(\zeta) = 0
\end{equation}
\end{itemize}
\end{theorem}
Непосредственно применяя эти теоремы, и пользуясь тем, что нашим вспомогательным пространством является $\bbC^2$, получаем, что мы можем воспользоваться данным упрощенным критерием для доказательства полноты резонансных состояний в квантовом графе.

\section{Критерий сходимости}

Так как $\det S$ является скалярной внутренней функцией \cite{nikol2012treatise}, мы можем воспользоваться критерием \ref{eq:blaschke} для исследования полноты системы резонантных состояний. В пространстве единичного диска он будем выглядеть как:
\begin{equation}\label{eq:crit_cayley}
\lim\limits_{r = 1} \int\limits_{\abs{\zeta} = r} \log \abs{\det S(\zeta)} d \zeta = 0
\end{equation}

Так как $S$-матрица естественным образом определена на комплексной плоскости, иногда бывает удобнее работать с этим признаком в верхней полуплоскости $\bbH$, а не на единичном диске. Заменим в (\ref{eq:crit_cayley}) переменную, применив к интегралу преобразование Кэли (\ref{eq:cayley}) к дифференциалу и области интегрирования, получив:

\begin{equation}\label{eq:crit}
\lim\limits_{r \to 1} \int\limits_{C_r} \ln \abs{\det S(k)} \frac{2 \iu}{(k + i)^2} dk = 0
\end{equation}
, где $C_r$ — образ $\abs{\zeta} = r$ относительно обратного преобразования Кэли. Для удобства вычислений параметризуем $C_r$ как $C_r = \{R(r) \eexp{\iu t} + \iu C(r) \mid t \in [0, 2 \pi)\}$ (см. \ref{eq:c_and_r}). Для краткости, обозначим:

\[
s(k) = \abs{\det S(k)}
\]
, и, откинув константы, не влияющие на сходимость/расходимость интеграла, получаем финальную форму критерия, которой и будем пользоваться в дальнейшем:

\begin{equation}\label{eq:critp}
\lim\limits_{r \to 1} \int\limits_{0}^{2 \pi} \ln s(R(r) \eexp{\iu t} + \iu C(r)) \frac{R}{(R(r) \eexp{\iu t} + \iu C(r) + i)^2} dt = 0
\end{equation}
\mtodo{$C + R = \Im w(r) = \frac{1 + r}{1 - r}$, $C - R = \Im w(-r) = \frac{1 - r}{1 + r}$}