\chapter{Предварительные сведения}

\section{Различные обозначения}
\begin{itemize}
\item $\bbC$ — комплексная плоскость, $\bbC = \{ x + \iu y \mid x, y \in \bbR \}$ 
\item $\bbH$ — верхняя комплексная полуплоскость (англ. upper half-plane), $\bbH = \{ x + \iu y \mid y > 0, x, y \in \bbR \}$
\item $\bbD$ — комплексный единичный круг (англ. unit disk), $\bbD = \{ z \mid \abs{z} < 1 \}$
\item $\bbT$ — комплексная единичная окружность (англ. unit circle), $\bbT = \partial \bbD =  \{z \mid \abs{z} = 1 \}$
\item $z$ используется в качестве переменной, обозрачающей точку в верхней комплексной полуплоскости $\bbH$
\item $\zeta$ используется в качестве переменной, обозрачающей точку на комплексном единичном диске $\bbD$
\end{itemize}

\section{Уравнение Шредингера}
\todo{рассеиватель}

\todo{резонаторы}

\section{S-матрица}

\todo{картиночку с полюсами и нулями??}

Рассмотрим локализованный одномерный потенциальный барьер или резонатор. Пусть на барьер слева и справа направлены частицы с волновым вектором $k$ (который будет скаляром в одномерном случае). Слева и справа от резонатора частицы ведут себя как свободные, соответственно, в общем виде их волновые функции имеют следующий вид:

\begin{equation}
\begin{aligned}
   \psi_L(x) &= A \eexp{\iu k x} + B \eexp{-\iu k x}
\\ \psi_R(x) &= C \eexp{\iu k x} + D \eexp{-\iu k x}
\end{aligned}
\end{equation}

S-матрица (матрица рассеяния, англ. scattering matrix) выражает зависимость исходящего состояния от входящего:
\[
\begin{pmatrix} B \\ C \end{pmatrix} = S \begin{pmatrix} A \\ D \end{pmatrix}
\]
, и полностью характеризует рассеивающие свойства потенциального барьера.

$S$-матрица обладает множествоминтересных свойств \cite[стр. 75]{perelomov1998quantum}, позволяющих анализировать рассеяние в квантовой системе: к примеру, $S$-матрица как функция комплексного аргумента, аналитична в верхней комплексной полуплоскости. 

\todo{написать про полюса и нули?}

\todo{симметричный/несимметричный случай}

\section{Квантовые графы}
Квантовый граф — широко используемая модель наносистемы \cite{kuchment2002graph, lobanov2013genetic, brown2010analysis}. 
\todo{с википедии достать?}
% https://en.wikipedia.org/wiki/Quantum_graph#Quantum_graphs
\mtodo{там по ссылке что-то очень интересное про S-матрицу}

\todo{написать что-нибудь. Про оператор, определнный на графе?}
Если граф $\Gamma$ состоит только из конечного числа ребер конечной длины, его гамильтониан имеет чисто дискретный спектр, и его собственные функции образуют полную систему в $\mcL_2(\Gamma)$. Если же граф $\Gamma$ представляет из себя резонатор $\Omega$ с полубесконечными ребрами, в спектре будет присутствовать непрерывная часть и резонансы, индуцированные собственными числами гамильтониана резонатора $\mcH_\Omega$. Резонансные состояния оператора Шредингера не принадлежат пространству $\mcL_2(\Gamma)$, однако, при сужении их на конечный домен $\Omega$, становятся квадратично интегрируемыми и лежат в пространстве $\mcL_2(\Omega)$. Для многих приложений интересно знать, формируют ли полную систему резонанстные состояния графа $\Gamma$ в пространстве $\mcL_2(\Omega)$ резонатора.

\section{Преобразование Кэли}

Прямое преобразование Кэли (англ. Cayley transform) конформно отображает $\bbH$ в $\bbD$:
\begin{equation}\label{eq:cayley}
W(z) = \frac{z - \iu}{z + \iu}
\end{equation}
, обратное преобразование Кэли аналогично отображает $\bbD$ в $\bbH$:
\begin{equation}\label{eq:cayley_inverse}
w(\zeta) = \iu \frac{1 + \zeta}{1 - \zeta}
\end{equation}

Важным свойством преобразование Кэли является инъективное отображение $\bbR$ в единичную окружность $\bbT$.

Так как преобразование Кэли является трансформацией Мебиуса, оно сохраняет окружности. В частности, окружность с радиусом $r$ в нуле, под действием обратного преобразование Кэли перейдет в окружность с центром в $C(r)$ и радиусом $R(r)$, где:

\begin{equation}\label{eq:c_and_r}
\begin{aligned}
   C(r) &= \Im \frac{w(r) + w(-r)}{2}
\\ R(r) &= \Im \frac{w(r) - w(-r)}{2}
\end{aligned}
\end{equation}

Легко заметить, что при стремлении $r$ к $1$, $R(r)$ стремится к бесконечности, а $C(r)$ стремится к $R(r)$, что естественно так как в пределе комплексная единичная окнужность отображается в вещественную ось.

\todo{использовать шаблоны определений, теоремы и т.п.}

\section{Внутренние функции и произведение Бляшке}
Для скалярных внутренних функций $\phi$ определенных на комплексном единичном диске $\bbD$, существует критерий отстутствия сингулярного множителя \cite[стр. 99]{nikol2012treatise}:
\[
\lim\limits_{r = 1} \int\limits_{\bbT} \log \abs{\phi(r \zeta)} d m(\zeta) = 0
\]

Так как $\det S$ является скалярной внутренней функцией, мы можем воспользоваться данным критерием для исследования полноты системы резонантных состояний.

\todo{мутно}


\subsection{Скалярное произведение}

В данной работе используется \textbf{физическая} нотация. В частности, это означает, что скалярное произведение в $L^2(E)$ определено как $\ip{f}{g} = \int\limits_E \cconj{f(\vb{x})} g(\vb{x}) \dd \vb{x}$.

\section{Терминология теории рассеяния}
Состояния рассеяния (англ. scattering states) — решения уравнения Шредингера, соответствующие непрерывному спектру, и не лежащие в $L^2$. Также частно называется каналом рассеяния (англ. scattering channel).

Мода (англ. mode) — связанная часть состояния рассеяния. О них обычно говорят в контексте волноводов, имеющих двумерную или трехмерную геометрию, и допускающих только связанные состояния в поперечном направлении. К примеру, в волноводе с конфигурацией $\Omega = [-\infty, \infty] \times [0, H]$ допустимыми поперечными модами являются $\psi_n(y) = \sqrt{\frac{2}{H}} \sin(\frac{\pi n}{H} y)$, $n \in \bbN^+$. Если поперечной частью волны является мода $n$, говорят, что «волна распространяется на $n$-й моде».

Открытый канал (англ. open channel) — канал рассеяния, на котором волна потенциально может распространяться при данной энергии $E$. Закрытый канал (англ. closed channel) — канал рассеяния, на котором волна не может распространяться при данной энергии $E$. Для вышеупомянутого примера двумерного волновода при любой энергии $E$ всегда будут открыты каналы $\{n \mid n \in \bbN^+, \left(\frac{\pi n}{H}\right)^2 < E \}$, которых, очевидно, конечное число.

\section{Критерий сходимости}

Так как $\det S$ является внутренней функцией \todo{REF}, \todo{блаблабла} критерий полноты в пространстве единичного диска выглядит как

\begin{equation}\label{eq:crit_cayley}
\lim\limits_{r = 1} \int\limits_{\abs{\zeta} = r} \log \abs{\det S(\zeta)} d \zeta = 0
\end{equation}

Так как $S$-матрица естественным образом определена на комплексной плоскости, иногда бывает удобнее работать с этим признаком в верхней полуплоскости $\bbH$, а не на единичном диске. Заменим в (\ref{eq:crit_cayley}) переменную, применив к интегралу преобразование Кэли (\ref{eq:cayley}) к дифференциалу и области интегрирования, получив:

\begin{equation}\label{eq:crit}
\lim\limits_{r \to 1} \int\limits_{C_r} \ln \abs{\det S(k)} \frac{2 \iu}{(k + i)^2} dk = 0
\end{equation}
, где $C_r$ — образ $\abs{\zeta} = r$ относительно обратного преобразования Кэли. Для удобства вычислений параметризуем $C_r$ как $C_r = \{R(r) \eexp{\iu t} + \iu C(r) \mid t \in [0, 2 \pi)\}$ (см. \ref{eq:c_and_r}). Для краткости, обозначим:

\[
s(k) = \abs{\det S(k)}
\]
, и после откидывания констант которые не влияют на сходимость/расходимость интеграла, получаем финальную форму критерия, которй и будем пользоваться в дальнейшем:

\begin{equation}\label{eq:critp}
\lim\limits_{r \to 1} \int\limits_{0}^{2 \pi} \ln s(R(r) \eexp{\iu t} + \iu C(r)) \frac{R}{(R(r) \eexp{\iu t} + \iu C(r) + i)^2} dt = 0
\end{equation}

\todo{$C + R = \Im w(r) = \frac{1 + r}{1 - r}$, $C - R = \Im w(-r) = \frac{1 - r}{1 + r}$}