\chapter{Предыдущие исследования данной задачи}\label{chap:prev_res}
\todo{какое-нибудь имя получше?}

\section{Модель резонатора типа «отрезок»}

\begin{figure}[!htb]
\centering
\begin{tikzpicture}[scale=1.1]
\newcommand{\Wglen}{10.0}; % waveguide length

\coordinate (LLL) at (-\Wglen - 1, 0);
\coordinate (RRR) at ( \Wglen + 1, 0);
\coordinate (LL)  at (-\Wglen, 0);
\coordinate (RR)  at ( \Wglen, 0);
% change these to take resonator length into account
\coordinate (L)   at (0, 0);
\coordinate (R)   at (0, 0);
%

\coordinate (U) at (0, 3); % upper point of resonator

% waveguide
\draw[ultra thick, dotted] (LLL) -- (LL);
\draw[ultra thick] (LL)--(L);
\draw[ultra thick] (R)--(RR);
\draw[ultra thick, dotted] (RR) -- (RRR);
%

\draw[<-] (-\Wglen + 1, 2) -- (-\Wglen + 5, 2) node [midway, above] {$R e^{-\iu k x}$};
\draw[->] (-\Wglen + 1, 0.5) -- (-\Wglen + 5, 0.5) node [midway, above] {$e^{\iu k x}$};
\draw[->] (\Wglen - 5, 0.5) -- (\Wglen - 1, 0.5) node [midway, above] {$T e^{\iu k x}$};

\draw[ultra thick] node [below] {$V$} (0, 0) -- node [left] {$\Omega$} (U);
\end{tikzpicture}
\caption{Одномерный резонатор-отрезок с условием Дирихле на краю резонатора.}
\end{figure}
\todo{ссылка на работу}

\section{Модель резонатора типа «квантовая точка»}

\begin{figure}[!htb]
\centering
\begin{tikzpicture}[scale=1.1]
\newcommand{\Wglen}{6.0}; % waveguide length
\newcommand{\Warrlen}{2.0}; % wave arrow length
\newcommand{\Reslen}{2.0}; % resonator length

\coordinate (LLL) at (-\Wglen - 1, 0);
\coordinate (RRR) at ( \Wglen + 1, 0);
\coordinate (LL)  at (-\Wglen, 0);
\coordinate (RR)  at ( \Wglen, 0);
\coordinate (L)   at (-\Reslen, 0);
\coordinate (R)   at ( \Reslen, 0);
%

\coordinate (U) at (0, 3); % upper point of resonator

% waveguide
\draw[ultra thick, dotted] (LLL) -- (LL);
\draw[ultra thick] (LL)--(L);
\draw[ultra thick] (R)--(RR);
\draw[ultra thick, dotted] (RR) -- (RRR);
%

\draw[<-, thick] (-\Wglen, 1.5) -- (-\Wglen + \Warrlen, 1.5) node [midway, above] {\large $R e^{-\iu k x}$};
\draw[->, thick] (-\Wglen, 0.8) -- (-\Wglen + \Warrlen, 0.8) node [midway, above] {\large $e^{\iu k x}$};
\draw[->, thick] (\Wglen - \Warrlen, 0.5) -- (\Wglen, 0.5)   node [midway, above] {\large $T e^{\iu k x}$};

\draw (LL) node [below] {\large $\Omega_L$};
\draw (RR) node [below] {\large $\Omega_R$};

\draw[ultra thick] (0, 2) node [above] {\Large $\Omega$};

\draw (-2,0) arc (180:360:2cm and 0.5cm);
\draw[dashed] (-2,0) arc (180:0:2cm and 0.5cm);
\draw (0,2) arc (90:270:0.5cm and 2cm);
\draw[dashed] (0,2) arc (90:-90:0.5cm and 2cm);
\draw (0,0) circle (2cm);
\shade[ball color=blue!10!white,opacity=0.20] (0,0) circle (2cm);
\end{tikzpicture}
\caption{Трехмерный резонатор (квантовая точка).}
\end{figure}
\todo{ссылка на работу}