\documentclass[12pt, a4paper, landscape]{article}

\usepackage[utf8]{inputenc}
\usepackage[T2A]{fontenc}
\usepackage[russian, english]{babel}
\usepackage{enumerate}
\usepackage{comment}
\usepackage{cite}
\usepackage{mathrsfs}
\usepackage{fullpage}
\usepackage{hyperref} % \url{url} \href{url}{displayed text}
\usepackage{tabularx}
\usepackage{graphicx}
\usepackage{indentfirst}
\usepackage{bm}
\usepackage{braket}

\usepackage{csquotes}
\MakeOuterQuote{"}
\usepackage{xcolor} % \textcolor
\usepackage{mathtools} % xmapsto, \Vector
\usepackage{comment}
\usepackage{soul} % text striking
\usepackage{amsmath,amsfonts,amssymb,amsthm,amscd,amsxtra}
\usepackage{physics}
\usepackage[sharp]{easylist}

\newcommand{\hilb}[1]{\mathcal{H}_{#1}}
\newcommand{\cconj}[1]{\overline{#1}}
\newcommand{\hank}[1]{H_{#1}^{(1)}}

\newcommand{\mcF}{\mathcal{F}}
\newcommand{\mcH}{\mathcal{H}} % ???
\newcommand{\mcI}{\mathcal{I}}
\newcommand{\mcL}{\mathcal{L}} % L2 space
\newcommand{\mcO}{\mathcal{O}} % Big O
\newcommand{\mcP}{\mathcal{P}}


\newcommand{\bbC}{\mathbb{C}} % complex plane
\newcommand{\bbD}{\mathbb{D}} % complex unit disk
\newcommand{\bbN}{\mathbb{N}}
\newcommand{\bbK}{\mathbb{K}}
\newcommand{\bbR}{\mathbb{R}}
\newcommand{\bbT}{\mathbb{T}} % complex unit circle
\newcommand{\bbZ}{\mathbb{Z}}

\newcommand{\eqdef}{\overset{\mathrm{def}}{=\joinrel=}}

\DeclareMathOperator{\dom}{dom}
\DeclareMathOperator{\ran}{Ran}
\DeclareMathOperator{\rng}{rng}

\newcommand{\todo}[1]{\textcolor{red}{{\large TODO: #1}}}

\newenvironment{elist}{\begin{easylist}[enumerate]}{\end{easylist}}
\newenvironment{ilist}{\begin{easylist}[itemize]}{\end{easylist}}

\newcommand{\myspecial}[1]{\mathrm{#1}}

% imaginary unit
\newcommand{\iu}{{i\mkern1mu}}


\newcommand{\ipcdot}{\ip{\cdot}{\cdot}}
\newcommand{\iip}[2]{[#1,#2]}
\newcommand{\iipcdot}{\iip{\cdot}{\cdot}}

\newcommand{\dsum}{\oplus}
\newcommand{\ddiff}{\ominus}
% indefinite direct sum
\newcommand{\idsum}{[+]}
\newcommand{\iddiff}{[-]}

\DeclarePairedDelimiter{\Vector}{\lparen}{\rparen}

\newcommand{\tit}{\textit}
\newcommand{\cls}{\overline}
\newcommand{\eps}{\varepsilon}


\newcommand{\argmin}{\operatornamewithlimits{argmin}}
\newcommand{\argmax}{\operatornamewithlimits{argmax}}

\renewcommand{\Re}{\operatorname{Re}}
\renewcommand{\Im}{\operatorname{Im}}
\renewcommand{\phi}{\varphi} % TODO is there a prettier way to do that

\newcommand{\eexp}[1]{e^{#1}}

\DeclareMathOperator\atanh{atanh}

% ???
% \newcommand{\abs}[1]{\left| #1 \right|}
% \newcommand{\norm}[1]{\left\lVert #1 \right\rVert}
\newcommand*\Eval[3]{\left.#1\right\rvert_{#2}^{#3}}

% \DeclareMathSizes{12}{19}{12}{12}
% TODO MORE TO TEMPLATE PREAMBLE!
\DeclareMathOperator\atanh{atanh}
\newcommand{\eexp}[1]{e^{#1}}
% \newcommand{\iu}{{i\mkern1mu}}
\renewcommand{\Re}{\operatorname{Re}}
\renewcommand{\Im}{\operatorname{Im}}
\renewcommand{\phi}{\varphi}



\begin{document}

Consider case $a = b = 0$ (no coupling).

Then, 
\[
\det S = \frac{2 i \, W \cos\left(k\right) - {\left(W^{2} + 1\right)} \sin\left(k\right)}{-2 i \, W \cos\left(k\right) - {\left(W^{2} + 1\right)} \sin\left(k\right)}
\]

Define $k = u + v \iu$. 


\[
\frac{W {\left(a + b\right)} k \cos\left(k\right) + 2 i \, W k^{2} \cos\left(k\right) + {\left(i \, a + i \, b\right)} k \sin\left(k\right) - {\left({\left(W^{2} + 1\right)} k^{2} - a b\right)} \sin\left(k\right)}{W {\left(a + b\right)} k \cos\left(k\right) - 2 i \, W k^{2} \cos\left(k\right) - {\left(i \, a + i \, b\right)} k \sin\left(k\right) - {\left({\left(W^{2} + 1\right)} k^{2} - a b\right)} \sin\left(k\right)}
\]

Absolute value, squared:
\[
\frac{{\left(\sin\left(u\right)^{2} + \sinh\left(v\right)^{2}\right)} W^{4} - 4 \, W^{3} \cosh\left(v\right) \sinh\left(v\right) - 2 \, {\left(\sin\left(u\right)^{2} - 3 \, \sinh\left(v\right)^{2} - 2\right)} W^{2} - 4 \, W \cosh\left(v\right) \sinh\left(v\right) + \sin\left(u\right)^{2} + \sinh\left(v\right)^{2}}{{\left(\sin\left(u\right)^{2} + \sinh\left(v\right)^{2}\right)} W^{4} + 4 \, W^{3} \cosh\left(v\right) \sinh\left(v\right) - 2 \, {\left(\sin\left(u\right)^{2} - 3 \, \sinh\left(v\right)^{2} - 2\right)} W^{2} + 4 \, W \cosh\left(v\right) \sinh\left(v\right) + \sin\left(u\right)^{2} + \sinh\left(v\right)^{2}}
\]

Replace $\sin(u)$ by minimum and maximum values where appropriate: 

\[
\frac{{\left(1 + \sinh\left(v\right)^{2}\right)} W^{4} - 4 \, W^{3} \cosh\left(v\right) \sinh\left(v\right) - 2 \, {\left(0 - 3 \, \sinh\left(v\right)^{2} - 2\right)} W^{2} - 4 \, W \cosh\left(v\right) \sinh\left(v\right) + 1 + \sinh\left(v\right)^{2}}{{\left(0 + \sinh\left(v\right)^{2}\right)} W^{4} + 4 \, W^{3} \cosh\left(v\right) \sinh\left(v\right) - 2 \, {\left(1 - 3 \, \sinh\left(v\right)^{2} - 2\right)} W^{2} + 4 \, W \cosh\left(v\right) \sinh\left(v\right) + 0 + \sinh\left(v\right)^{2}}
\]


Divide everything by $\cosh^2(v)$ and use the estimate $0 \le \frac{1}{\cosh^2(v)} \le 1$: TODO ERROR tanh

\[
\frac
{{\left(1 + 1\right)} W^{4} - 4 \, W^{3} \tanh\left(v\right) - 2 \, {\left(0 - 3 - 2\right)} W^{2} - 4 \, W \tanh\left(v\right) + 1 + \tanh\left(v\right)^{2}}
{{\left(0 + 1\right)} W^{4} + 4 \, W^{3} \tanh\left(v\right) - 2 \, {\left(1 - 3 - 0\right)} W^{2} + 4 \, W \tanh\left(v\right) + 0 + \tanh\left(v\right)^{2}}
\]

Now, estimage $0 \le \tanh(v) \le 1$:

\[
\frac
{{\left(1 + 1\right)} W^{4} - 4 \, W^{3} 1 - 2 \, {\left(0 - 3 - 2\right)} W^{2} - 4 \, W 0 + 1 + 1}
{{\left(0 + 1\right)} W^{4} + 4 \, W^{3} 0 - 2 \, {\left(1 - 3 - 0\right)} W^{2} + 4 \, W 0 + 0 + 0}
\]

Since $W$ is constant for every specific instance of scattering problem, the upper bound is constant as well.


Lower bound:

Absolute value, squared:
\[
\frac{{\left(\sin\left(u\right)^{2} + \sinh\left(v\right)^{2}\right)} W^{4} - 4 \, W^{3} \cosh\left(v\right) \sinh\left(v\right) - 2 \, {\left(\sin\left(u\right)^{2} - 3 \, \sinh\left(v\right)^{2} - 2\right)} W^{2} - 4 \, W \cosh\left(v\right) \sinh\left(v\right) + \sin\left(u\right)^{2} + \sinh\left(v\right)^{2}}{{\left(\sin\left(u\right)^{2} + \sinh\left(v\right)^{2}\right)} W^{4} + 4 \, W^{3} \cosh\left(v\right) \sinh\left(v\right) - 2 \, {\left(\sin\left(u\right)^{2} - 3 \, \sinh\left(v\right)^{2} - 2\right)} W^{2} + 4 \, W \cosh\left(v\right) \sinh\left(v\right) + \sin\left(u\right)^{2} + \sinh\left(v\right)^{2}}
\]

Replace $\sin(u)$ by minimum and maximum values where appropriate: 

\[
\frac
{{\left(0 + \sinh\left(v\right)^{2}\right)} W^{4} - 4 \, W^{3} \cosh\left(v\right) \sinh\left(v\right) - 2 \, {\left(1 - 3 \, \sinh\left(v\right)^{2} - 2\right)} W^{2} - 4 \, W \cosh\left(v\right) \sinh\left(v\right) + 0 + \sinh\left(v\right)^{2}}
{{\left(1 + \sinh\left(v\right)^{2}\right)} W^{4} + 4 \, W^{3} \cosh\left(v\right) \sinh\left(v\right) - 2 \, {\left(0 - 3 \, \sinh\left(v\right)^{2} - 2\right)} W^{2} + 4 \, W \cosh\left(v\right) \sinh\left(v\right) + 1 + \sinh\left(v\right)^{2}}
\]

Divide everything by $\sinh^2(v)$ and use the estimate TODO!!! $0 \le \frac{1}{\sinh^2(v)} \le C_1$:

\[
\frac
{{\left(0   + 1\right)} W^{4} - 4 \, W^{3} \coth\left(v\right) - 2 \, {\left(C_1 - 3 - 2 \cdot 0  \right)} W^{2} - 4 \, W \coth\left(v\right) + 0   + 1}
{{\left(C_1 + 1\right)} W^{4} + 4 \, W^{3} \coth\left(v\right) - 2 \, {\left(0   - 3 - 2       C_1\right)} W^{2} + 4 \, W \coth\left(v\right) + C_1 + 1}
\]


Now, estimage $1 \le \coth(v) \le D_1$:

\[
\frac
{{\left(0   + 1\right)} W^{4} - 4 \, W^{3} D_1 - 2 \, {\left(C_1 - 3 - 2 \cdot 0  \right)} W^{2} - 4 \, W D_1 + 0   + 1}
{{\left(C_1 + 1\right)} W^{4} + 4 \, W^{3} D_1 - 2 \, {\left(0   - 3 - 2       C_1\right)} W^{2} + 4 \, W D_1 + C_1 + 1}
\]

We observe two things: 1) the value of the polynomial in the denominator is always greater than zero
2) for $W \ge 2$ we can always find $v$ big enough so that constants $C_1(v)$ and $D_1(v)$ are small enough to make the polynomial in the numerator greater than zero. Hence, we got a positive lower bound on the absolute value of $\det S$ as well. 

\[
0 < B_l \le \abs{\det S} \le B_u \implies \ln B_l \le \ln \abs{\det S} \le \ln B_u \implies \abs{\ln \abs{\det S}} \le \max(\abs{\ln B_l}, \abs{\ln B_u})
\]

Which means that the integral over the upper arc converges to zero as $R \to \infty$.



\[
\frac{{\left({\left(u^{2} \sin\left(u\right) + 2 i \, u v \sin\left(u\right) - v^{2} \sin\left(u\right)\right)} \cosh\left(v\right) - {\left(-i \, u^{2} \cos\left(u\right) + 2 \, u v \cos\left(u\right) + i \, v^{2} \cos\left(u\right)\right)} \sinh\left(v\right)\right)} W^{2} - {\left({\left({\left(i \, a + i \, b - 4 \, u\right)} v \cos\left(u\right) - 2 i \, v^{2} \cos\left(u\right) + {\left({\left(a + b\right)} u + 2 i \, u^{2}\right)} \cos\left(u\right)\right)} \cosh\left(v\right) + {\left({\left(a + b + 4 i \, u\right)} v \sin\left(u\right) - 2 \, v^{2} \sin\left(u\right) + {\left({\left(-i \, a - i \, b\right)} u + 2 \, u^{2}\right)} \sin\left(u\right)\right)} \sinh\left(v\right)\right)} W + {\left({\left(a + b + 2 i \, u\right)} v \sin\left(u\right) - v^{2} \sin\left(u\right) - {\left(a b + {\left(i \, a + i \, b\right)} u - u^{2}\right)} \sin\left(u\right)\right)} \cosh\left(v\right) - {\left({\left(-i \, a - i \, b + 2 \, u\right)} v \cos\left(u\right) + i \, v^{2} \cos\left(u\right) + {\left(i \, a b - {\left(a + b\right)} u - i \, u^{2}\right)} \cos\left(u\right)\right)} \sinh\left(v\right)}{{\left({\left(u^{2} \sin\left(u\right) + 2 i \, u v \sin\left(u\right) - v^{2} \sin\left(u\right)\right)} \cosh\left(v\right) - {\left(-i \, u^{2} \cos\left(u\right) + 2 \, u v \cos\left(u\right) + i \, v^{2} \cos\left(u\right)\right)} \sinh\left(v\right)\right)} W^{2} - {\left({\left({\left(i \, a + i \, b + 4 \, u\right)} v \cos\left(u\right) + 2 i \, v^{2} \cos\left(u\right) + {\left({\left(a + b\right)} u - 2 i \, u^{2}\right)} \cos\left(u\right)\right)} \cosh\left(v\right) + {\left({\left(a + b - 4 i \, u\right)} v \sin\left(u\right) + 2 \, v^{2} \sin\left(u\right) + {\left({\left(-i \, a - i \, b\right)} u - 2 \, u^{2}\right)} \sin\left(u\right)\right)} \sinh\left(v\right)\right)} W - {\left({\left(a + b - 2 i \, u\right)} v \sin\left(u\right) + v^{2} \sin\left(u\right) + {\left(a b + {\left(-i \, a - i \, b\right)} u - u^{2}\right)} \sin\left(u\right)\right)} \cosh\left(v\right) - {\left({\left(i \, a + i \, b + 2 \, u\right)} v \cos\left(u\right) + i \, v^{2} \cos\left(u\right) + {\left(i \, a b + {\left(a + b\right)} u - i \, u^{2}\right)} \cos\left(u\right)\right)} \sinh\left(v\right)}
\]

We compute the upper bound. First, we vary terms dependent on $u$:


Let's take a look at the behaviour of the expression above as $v \to \infty$:

Divide numerator and denominator by $\cosh(v)$ (we are allowed to do that since $v > 0$).

\[
\det S = - \frac{2 i \, W \cos\left(k\right) - {\left(W^{2} + 1\right)} \sin\left(k\right)}{2 i \, W \cos\left(k\right) + {\left(W^{2} + 1\right)} \sin\left(k\right)}
\]

\subsection*{0. Prove that abs(S) is bounded as 1 in upper complex plane, therefore, ln is less than zero}
\todo{I guess it holds in general? }

Now, we give a lower bound for $\abs{\det S}$ which induces an upper bound for $\ln \abs{\det S}$. We prove that integral over this upper bound converges.

\subsection*{1. Notice that function is periodic w.r.t. real part.}
\todo{Proof}

All zeroes are at $\Im k = \atanh \frac{2 W}{W^2 + 1}$. Define $Z = \atanh \frac{2 W}{W^2 + 1}$

\todo{meh}

% TODO WHERE DOES THIS abs come from?
\subsection*{2. Notice that for any $x$, $\abs{S(x + \iu y)} \le \abs{S(\iu y)}$}
\todo{Proof}

Hence, we replace function with

\[
f(x + \iu y)
 = \frac{(W^2 + 1) \sinh y - 2 W \cosh y}{(W^2 + 1) \sinh y + 2 W \cosh y}
 = \frac{\tanh y - \frac{2 W}{W^2 + 1}}{\tanh y + \frac{2 W}{W^2 + 1}}
\]

So, now it has a singularity along the line $\Im k = \atanh \frac{2 W}{W^2 + 1}$

\begin{figure}[!htb]
\includegraphics{est1.png}
\end{figure}

\subsection*{3. Split the integral in two parts: below the singularity and above. Below:}
Replace the function by a smaller one. First, we get rid of the denominator: we use the fact that $\tanh y + \frac{2 W}{W^2 + 1}$ is positive and strictly increases as $y$ goes from $0$ to $\atanh \frac{2 W}{W^2 + 1}$, and its maximum is actually $\frac{4 W}{W^2 + 1}$. We replace the denominator:

\[
f(x + \iu y)
 = \frac{\tanh y - \frac{2 W}{W^2 + 1}}{2 \frac{2 W}{W^2 + 1}}
 = \frac{W^2 + 1}{4 W} \tanh y - \frac{1}{2}
\]

\subsection*{4. Notice that the function is convex left to its zero}
\todo{Proof}

Hence, we can give a lower bound to the function using its first derivative at $Z$:

% TODO we can elaborate here
\[
f(y) = h'(Z) (y - Z)
\]

\[
h'(Z) = \frac{{\left(W^{2} - 1\right)}^{2}}{4 \, {\left(W^{2} + 1\right)} W}
\]
% h'(W=2) = 9/40; h'(W=3) = 8/15

\begin{figure}[!htb]
\includegraphics{est_der.png}
\end{figure}

\subsection*{5. Ok, function is good enough to take the logarithm and integrate!}
We've got a logarithm of linear function. The suspicious part is that on the integration path, function under logarithm takes the value of $0$.

Integration goes from $\phi = -\frac{\pi}{2}$ to $\phi = -\frac{\pi}{2} + \acos \frac{R - Z}{R}$

\[
\ln f(\Im (R \eexp{\iu t} + R \iu)) R \iu \eexp{\iu t} \frac{1}{(R \eexp{\iu t} - 1)^2} = 
\ln f(R (\sin t + 1)) R \iu \eexp{\iu t} \frac{1}{(R \eexp{\iu t} - 1)^2} = 
\]

\todo{Use the fact that $\sin t \le \frac{2 \pi}{t}$?}

\todo{extract the constant derivative multiplier?}

\subsection*{Second part of integral}
Since the function is concave, we can take any $y_0 > Z$ and bound $\abs{\det S}$ from below as piecewise linear function:

\[
f(y) =
\begin{cases}
\text{linear function} &, Z \le y \le y_0 \\
f_p(y_0),              &, y > y_0
\end{cases}
\]

\todo{Linear part proof should be similar to first part of integral}

\todo{constant part is trivial?}

% % Волна слева от рассеивателя:
% \[
% \psi_L(x) = A e^{i k x} + B e^{-i k x}
% \]
% справа:
% \[
% \psi_R(x) = C e^{i k x} + D e^{-i k x}
% \]
% Ну и
% \[
% \begin{pmatrix}B \\ C \end{pmatrix} = \begin{pmatrix} S_{11} & S_{12} \\ S_{21} & S_{22} \end{pmatrix}\begin{pmatrix} A \\ D \end{pmatrix}
% \]


% Уравнения на рассеиватель: на непрерывность функций

% \[
% \psi_L(0) = \psi_1(0) = \psi_2(0)
% \]
% \[
% \psi_R(L) = \psi_1(L) = \psi_2(L)
% \]

% И условия на производную (для простоты, без скачка, $a=0$):

% \[
% -\psi'_L(0) + \psi'_1(0) + \psi'_2(0) = 0
% \]
% \[
% \psi'_R(L) - \psi'_1(L) - \psi'_2(L) = 0
% \]

% Для простоты дальше возьмем $L=1$; получаем систему:

% \[
% \begin{cases}
% A + B = Q_1 + Q_2 \\
% Q_1 e^{i k} + Q_2 e^{-i k} = C e^{i k} + D e^{-i k} \\
% A + B = W_1 + W_2 \\
% W_1 e^{i k} + W_2 e^{-i k} = C e^{i k} + D e^{-i k} \\
% -i A k + i B k + i Q_1 k - i Q_2 k + i W_1 k - i W_2 k = 0 \\
% i C k e^{i k} - i Q_1 k e^{i k} - i W_1 k e^{i k} - i D k e^{-i k} + i Q_2 k e^{-i k} + i W_2 k e^{-i k} = 0
% \end{cases}
% \]
% % A + B == Q1 + Q2
% % Q1*e^(I*k) + Q2*e^(-I*k) == C*e^(I*k) + D*e^(-I*k)
% % A + B == W1 + W2
% % W1*e^(I*k) + W2*e^(-I*k) == C*e^(I*k) + D*e^(-I*k)
% % -I*A*k + I*B*k + I*Q1*k - I*Q2*k + I*W1*k - I*W2*k == 0
% % I*C*k*e^(I*k) - I*Q1*k*e^(I*k) - I*W1*k*e^(I*k) - I*D*k*e^(-I*k) + I*Q2*k*e^(-I*k) + I*W2*k*e^(-I*k) == 0

% Решаем систему, получаем:
% \[
% \begin{cases}
% %  B: -(3*A*(e^(2*I*k) - 1) + 8*D)/(e^(2*I*k) - 9),
% %  C: -(3*D*(e^(2*I*k) - 1) + 8*A*e^(2*I*k))/(e^(4*I*k) - 9*e^(2*I*k))}
% B =                     & \frac{1}{e^{2 i k} - 9} ( -3 A (e^{2 i k} - 1) - 8 D ) \\ 
% C = \frac{1}{e^{2 i k}} & \frac{1}{e^{2 i k} - 9} ( -3 D (e^{2 i k} - 1) - 8 A e^{2 i k} ) 
% \end{cases}
% \]

% Отсюда сразу видно определитель $S$-матрицы:

% \[
% \det S = \frac{1}{e^{2 i k}} \left( \frac{1}{e^{2 i k} - 9} \frac{1}{e^{2 i k} - 9} \left( 9 (e^{2 i k} - 1)^2 - 64 e^{2 i k} \right) \right)
% \]

% \[
% \abs{\det S} = \abs{\frac{1}{e^{2 i k}} \left( \frac{1}{e^{2 i k} - 9} \frac{1}{e^{2 i k} - 9} \left( 9 (e^{2 i k} - 1)^2 - 64 e^{2 i k} \right) \right)} 
% \]
% \[
%              = \abs{\frac{1}{e^{2 i k}}}  \abs{\frac{1}{e^{2 i k} - 9} \frac{1}{e^{2 i k} - 9} \left( 9 (e^{2 i k} - 1)^2 - 64 e^{2 i k} \right)} 
% \]

% Логарифм :

% \[
% \ln \abs{\det S} = \ln \abs{\frac{1}{e^{2 i k}}} + \ln  \abs{\frac{1}{e^{2 i k} - 9} \frac{1}{e^{2 i k} - 9} \left( 9 (e^{2 i k} - 1)^2 - 64 e^{2 i k} \right)} 
% \]

% С правым слагаемым проблем нет, оно при интегрировании по контуру стремится к нулю. А вот с левым плохо:

% \[
% \ln \abs{\frac{1}{e^{2 i k}}} = 2 \Im k
% \]

% Эта штука при интегрировании не зануляется, получается константа, в конкретном случае вроде $\pi$, то есть критерий не выполняется, и система функций будет неполной. 


% f_inc(x) = A * exp(i * k * x) + B * exp(-i * k * x)
% f_out(x) = C * exp(i * k * x) + D * exp(-i * k * x)



% Consider the following scattering setup: two half-infitine wires $\Omega_L$, $\Omega_R$ and a scatterer. Scatterer is a bundle composed of $W$ wires of length TODO. \\

% We impose $\delta$-coupling boundary conditions on vertices.

% Boundary conditions: for all $i$:
% \[
% \psi_L(V_L) = \psi_i(V_L)
% \]
% \[
% \psi_i(V_R) = \psi_R(V_R)
% \]
% TODO delta coupling

% For the sake of simplicity, we TODO

% After solving the system, we get:

% \[
% \det S(k) = e^{-2 i L k} \frac{
%     W k (a + b + 2 i k) \cos(k L) + (a b + i (a + b) k - (W^2 + 1) k^2) \sin(k L)
% }{
%     W k (a + b - 2 i k) \cos(k L) + (a b - i (a + b) k - (W^2 + 1) k^2) \sin(k L)
% }
% \]


% % TODO Imag part
% \[
% \ln \abs{\det S(k)} = \ln \abs{e^{-2 i L k}} + \ln \abs{R} = 2 L \Im k + \ln \abs{R}
% \]

% Second part results in $0$.

% First part results exactly in $\pi$.

% We consider scattering from the left. 

% Due to the symmetry of the system TODO


% num = W * k * ((a + b) + 2 * i * k) * cos(k * L) * exp(-i * L * k) + (a * b + i * (a + b) * k - coeff * k**2) * exp(-i * L * k) * sin(k * L)
% den = W * k * ((a + b) - 2 * i * k) * cos(k * L) * exp( i * L * k) + (a * b - i * (a + b) * k - coeff * k**2) * exp( i * L * k) * sin(k * L)
% return num / den

% abs(num/den) = abs(num)/abs(den).

\end{document}
