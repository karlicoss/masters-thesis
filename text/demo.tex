\documentclass[12pt, a4paper, landscape]{article}

\usepackage[utf8]{inputenc}
\usepackage[T2A]{fontenc}
\usepackage[russian, english]{babel}
\usepackage{enumerate}
\usepackage{comment}
\usepackage{cite}
\usepackage{mathrsfs}
\usepackage{fullpage}
\usepackage{hyperref} % \url{url} \href{url}{displayed text}
\usepackage{tabularx}
\usepackage{graphicx}
\usepackage{indentfirst}
\usepackage{bm}
\usepackage{braket}

\usepackage{csquotes}
\MakeOuterQuote{"}
\usepackage{xcolor} % \textcolor
\usepackage{mathtools} % xmapsto, \Vector
\usepackage{comment}
\usepackage{soul} % text striking
\usepackage{amsmath,amsfonts,amssymb,amsthm,amscd,amsxtra}
\usepackage{physics}
\usepackage[sharp]{easylist}

\newcommand{\hilb}[1]{\mathcal{H}_{#1}}
\newcommand{\cconj}[1]{\overline{#1}}
\newcommand{\hank}[1]{H_{#1}^{(1)}}

\newcommand{\mcF}{\mathcal{F}}
\newcommand{\mcH}{\mathcal{H}} % ???
\newcommand{\mcI}{\mathcal{I}}
\newcommand{\mcL}{\mathcal{L}} % L2 space
\newcommand{\mcO}{\mathcal{O}} % Big O
\newcommand{\mcP}{\mathcal{P}}


\newcommand{\bbC}{\mathbb{C}} % complex plane
\newcommand{\bbD}{\mathbb{D}} % complex unit disk
\newcommand{\bbH}{\mathbb{H}} % complex upper half-plane TODO add to template
\newcommand{\bbN}{\mathbb{N}}
\newcommand{\bbK}{\mathbb{K}}
\newcommand{\bbR}{\mathbb{R}}
\newcommand{\bbT}{\mathbb{T}} % complex unit circle
\newcommand{\bbZ}{\mathbb{Z}}

\newcommand{\eqdef}{\overset{\mathrm{def}}{=\joinrel=}}

\DeclareMathOperator{\dom}{dom}
\DeclareMathOperator{\ran}{Ran}
\DeclareMathOperator{\rng}{rng}

\newcommand{\todo}[1]{\textcolor{red}{{\large TODO: #1}}}

\newenvironment{elist}{\begin{easylist}[enumerate]}{\end{easylist}}
\newenvironment{ilist}{\begin{easylist}[itemize]}{\end{easylist}}

\newcommand{\myspecial}[1]{\mathrm{#1}}

% imaginary unit
\newcommand{\iu}{{i\mkern1mu}}


\newcommand{\ipcdot}{\ip{\cdot}{\cdot}}
\newcommand{\iip}[2]{[#1,#2]}
\newcommand{\iipcdot}{\iip{\cdot}{\cdot}}

\newcommand{\dsum}{\oplus}
\newcommand{\ddiff}{\ominus}
% indefinite direct sum
\newcommand{\idsum}{[+]}
\newcommand{\iddiff}{[-]}

\DeclarePairedDelimiter{\Vector}{\lparen}{\rparen}

\newcommand{\tit}{\textit}
\newcommand{\cls}{\overline}
\newcommand{\eps}{\varepsilon}


\newcommand{\argmin}{\operatornamewithlimits{argmin}}
\newcommand{\argmax}{\operatornamewithlimits{argmax}}

\renewcommand{\Re}{\operatorname{Re}}
\renewcommand{\Im}{\operatorname{Im}}
\renewcommand{\phi}{\varphi} % TODO is there a prettier way to do that

\newcommand{\eexp}[1]{e^{#1}}

\DeclareMathOperator\atanh{atanh}

% ???
% \newcommand{\abs}[1]{\left| #1 \right|}
% \newcommand{\norm}[1]{\left\lVert #1 \right\rVert}
\newcommand*\Eval[3]{\left.#1\right\rvert_{#2}^{#3}}

\DeclareMathSizes{12}{19}{12}{12}

\begin{document}

Волна слева от рассеивателя:
\[
\psi_L(x) = A e^{i k x} + B e^{-i k x}
\]
справа:
\[
\psi_R(x) = C e^{i k x} + D e^{-i k x}
\]
Ну и
\[
\begin{pmatrix}B \\ C \end{pmatrix} = \begin{pmatrix} S_{11} & S_{12} \\ S_{21} & S_{22} \end{pmatrix}\begin{pmatrix} A \\ D \end{pmatrix}
\]


Уравнения на рассеиватель: на непрерывность функций

\[
\psi_L(0) = \psi_1(0) = \psi_2(0)
\]
\[
\psi_R(L) = \psi_1(L) = \psi_2(L)
\]

И условия на производную (для простоты, без скачка, $a=0$):

\[
-\psi'_L(0) + \psi'_1(0) + \psi'_2(0) = 0
\]
\[
\psi'_R(L) - \psi'_1(L) - \psi'_2(L) = 0
\]

Для простоты дальше возьмем $L=1$; получаем систему:

\[
\begin{cases}
A + B = Q_1 + Q_2 \\
Q_1 e^{i k} + Q_2 e^{-i k} = C e^{i k} + D e^{-i k} \\
A + B = W_1 + W_2 \\
W_1 e^{i k} + W_2 e^{-i k} = C e^{i k} + D e^{-i k} \\
-i A k + i B k + i Q_1 k - i Q_2 k + i W_1 k - i W_2 k = 0 \\
i C k e^{i k} - i Q_1 k e^{i k} - i W_1 k e^{i k} - i D k e^{-i k} + i Q_2 k e^{-i k} + i W_2 k e^{-i k} = 0
\end{cases}
\]
% A + B == Q1 + Q2
% Q1*e^(I*k) + Q2*e^(-I*k) == C*e^(I*k) + D*e^(-I*k)
% A + B == W1 + W2
% W1*e^(I*k) + W2*e^(-I*k) == C*e^(I*k) + D*e^(-I*k)
% -I*A*k + I*B*k + I*Q1*k - I*Q2*k + I*W1*k - I*W2*k == 0
% I*C*k*e^(I*k) - I*Q1*k*e^(I*k) - I*W1*k*e^(I*k) - I*D*k*e^(-I*k) + I*Q2*k*e^(-I*k) + I*W2*k*e^(-I*k) == 0

Решаем систему, получаем:
\[
\begin{cases}
%  B: -(3*A*(e^(2*I*k) - 1) + 8*D)/(e^(2*I*k) - 9),
%  C: -(3*D*(e^(2*I*k) - 1) + 8*A*e^(2*I*k))/(e^(4*I*k) - 9*e^(2*I*k))}
B =                     & \frac{1}{e^{2 i k} - 9} ( -3 A (e^{2 i k} - 1) - 8 D ) \\ 
C = \frac{1}{e^{2 i k}} & \frac{1}{e^{2 i k} - 9} ( -3 D (e^{2 i k} - 1) - 8 A e^{2 i k} ) 
\end{cases}
\]

Отсюда сразу видно определитель $S$-матрицы:

\[
\det S = \frac{1}{e^{2 i k}} \left( \frac{1}{e^{2 i k} - 9} \frac{1}{e^{2 i k} - 9} \left( 9 (e^{2 i k} - 1)^2 - 64 e^{2 i k} \right) \right)
\]

\[
\abs{\det S} = \abs{\frac{1}{e^{2 i k}} \left( \frac{1}{e^{2 i k} - 9} \frac{1}{e^{2 i k} - 9} \left( 9 (e^{2 i k} - 1)^2 - 64 e^{2 i k} \right) \right)} 
\]
\[
             = \abs{\frac{1}{e^{2 i k}}}  \abs{\frac{1}{e^{2 i k} - 9} \frac{1}{e^{2 i k} - 9} \left( 9 (e^{2 i k} - 1)^2 - 64 e^{2 i k} \right)} 
\]

Логарифм :

\[
\ln \abs{\det S} = \ln \abs{\frac{1}{e^{2 i k}}} + \ln  \abs{\frac{1}{e^{2 i k} - 9} \frac{1}{e^{2 i k} - 9} \left( 9 (e^{2 i k} - 1)^2 - 64 e^{2 i k} \right)} 
\]

С правым слагаемым проблем нет, оно при интегрировании по контуру стремится к нулю. А вот с левым плохо:

\[
\ln \abs{\frac{1}{e^{2 i k}}} = 2 \Im k
\]

Эта штука при интегрировании не зануляется, получается константа, в конкретном случае вроде $\pi$, то есть критерий не выполняется, и система функций будет неполной. 


% f_inc(x) = A * exp(i * k * x) + B * exp(-i * k * x)
% f_out(x) = C * exp(i * k * x) + D * exp(-i * k * x)



% Consider the following scattering setup: two half-infitine wires $\Omega_L$, $\Omega_R$ and a scatterer. Scatterer is a bundle composed of $W$ wires of length TODO. \\

% We impose $\delta$-coupling boundary conditions on vertices.

% Boundary conditions: for all $i$:
% \[
% \psi_L(V_L) = \psi_i(V_L)
% \]
% \[
% \psi_i(V_R) = \psi_R(V_R)
% \]
% TODO delta coupling

% For the sake of simplicity, we TODO

% After solving the system, we get:

% \[
% \det S(k) = e^{-2 i L k} \frac{
%     W k (a + b + 2 i k) \cos(k L) + (a b + i (a + b) k - (W^2 + 1) k^2) \sin(k L)
% }{
%     W k (a + b - 2 i k) \cos(k L) + (a b - i (a + b) k - (W^2 + 1) k^2) \sin(k L)
% }
% \]


% % TODO Imag part
% \[
% \ln \abs{\det S(k)} = \ln \abs{e^{-2 i L k}} + \ln \abs{R} = 2 L \Im k + \ln \abs{R}
% \]

% Second part results in $0$.

% First part results exactly in $\pi$.

% We consider scattering from the left. 

% Due to the symmetry of the system TODO


% num = W * k * ((a + b) + 2 * i * k) * cos(k * L) * exp(-i * L * k) + (a * b + i * (a + b) * k - coeff * k**2) * exp(-i * L * k) * sin(k * L)
% den = W * k * ((a + b) - 2 * i * k) * cos(k * L) * exp( i * L * k) + (a * b - i * (a + b) * k - coeff * k**2) * exp( i * L * k) * sin(k * L)
% return num / den

% abs(num/den) = abs(num)/abs(den).

\end{document}
