\documentclass[a4paper]{report}

%uncomment to see the references
%\usepackage{showkeys}
\usepackage[T2A]{fontenc}

\usepackage[section]{algorithm} % ???
\usepackage{algorithmic} % ???
\usepackage[english,russian]{babel}

\usepackage[backend=biber,bibencoding=utf8,sorting=none,sortcites=true,bibstyle=sty/gost71,maxnames=99,citestyle=numeric-comp,babel=other]{biblatex
}

\defbibenvironment{bibliography}
  {\list
     {\printfield[labelnumberwidth]{labelnumber}.}
     {\setlength{\labelwidth}{2\labelnumberwidth}%
      \setlength{\leftmargin}{\labelwidth}%
      \setlength{\labelsep}{\biblabelsep}%
      \addtolength{\leftmargin}{\labelsep}%
      \setlength{\itemsep}{\bibitemsep}%
      \setlength{\parsep}{\bibparsep}}%
      \renewcommand*{\makelabel}[1]{\hss##1}}
  {\endlist}
  {\item}


\usepackage[utf8]{inputenc}
\usepackage{csquotes} % ????
%\usepackage{expdlist}
%\usepackage[nottoc,notbib]{tocbibind}
\usepackage[pdftex]{graphicx}
\graphicspath{{pic/}}
\usepackage{tikz}

\usepackage{mathtools} % xmapsto, \Vector
\usepackage{comment}
\usepackage{soul} % text striking
\usepackage{amsmath,amsfonts,amssymb,amsthm,amscd,amsxtra}
\usepackage{physics}
\usepackage[sharp]{easylist}
\usepackage{siunitx}
\usepackage{sty/kbordermatrix}
\sisetup{locale=US}


\usepackage{sty/dbl12}
%\usepackage{srcltx}
\usepackage{epsfig}
% \usepackage{verbatim}
\usepackage{sty/rac}
\usepackage[singlelinecheck=false]{caption}

\usepackage{xcolor, colortbl}
\definecolor{light-gray}{RGB}{230,230,230}
\definecolor{dkgreen}{RGB}{0,154,0}
\definecolor{gray}{RGB}{128,128,128}
\definecolor{mauve}{RGB}{149,0,210}
\definecolor{purpur}{RGB}{255,204,153}
%%%%%%%%%%%%%%%%%%%%%%%%%%%%%%%%%%%%%%%%%%%%%%%%%%%%%%%%%%%%%%%%%%%%%%%%%%%%%%

\captionsetup[figure]{justification=centering,   position=bottom, skip=0pt}
\captionsetup[table] {justification=raggedright, position=top,    skip=0pt}

% Redefine margins and other page formatting

\setlength{\oddsidemargin}{0.5in}

% Various theorem environments. All of the following have the same numbering
% system as theorem.

\theoremstyle{plain}
\newtheorem{theorem}{Теорема}
\newtheorem{prop}[theorem]{Утверждение}
\newtheorem{corollary}[theorem]{Следствие}
\newtheorem{lemma}[theorem]{Лемма}
\newtheorem{question}[theorem]{Вопрос}
\newtheorem{conjecture}[theorem]{Гипотеза}
\newtheorem{assumption}[theorem]{Предположение}

\theoremstyle{definition}
\newtheorem{definition}[theorem]{Определение}
\newtheorem{notation}[theorem]{Обозначение}
\newtheorem{condition}[theorem]{Условие}
\newtheorem{example}[theorem]{Пример}
%\newtheorem{algorithm}[theorem]{Алгоритм}
\floatname{algorithm}{Листинг}
\renewcommand{\algorithmicrequire}{\textbf{Вход:}}

\renewcommand{\proofname}{Доказательство}

%\newtheorem{introduction}[theorem]{Introduction}

\renewcommand{\proof}{\\\textbf{Доказательство.}~}
 
\newcommand{\hilb}[1]{\mathcal{H}_{#1}}
\newcommand{\cconj}[1]{\overline{#1}}
\newcommand{\hank}[1]{H_{#1}^{(1)}}

\newcommand{\mcF}{\mathcal{F}}
\newcommand{\mcH}{\mathcal{H}} % ???
\newcommand{\mcI}{\mathcal{I}}
\newcommand{\mcL}{\mathcal{L}} % L2 space
\newcommand{\mcO}{\mathcal{O}} % Big O
\newcommand{\mcP}{\mathcal{P}}


\newcommand{\bbC}{\mathbb{C}} % complex plane
\newcommand{\bbD}{\mathbb{D}} % complex unit disk
\newcommand{\bbN}{\mathbb{N}}
\newcommand{\bbK}{\mathbb{K}}
\newcommand{\bbR}{\mathbb{R}}
\newcommand{\bbT}{\mathbb{T}} % complex unit circle
\newcommand{\bbZ}{\mathbb{Z}}

\newcommand{\eqdef}{\overset{\mathrm{def}}{=\joinrel=}}

\DeclareMathOperator{\dom}{dom}
\DeclareMathOperator{\ran}{Ran}
\DeclareMathOperator{\rng}{rng}

\newcommand{\todo}[1]{\textcolor{red}{{\large TODO: #1}}}

\newenvironment{elist}{\begin{easylist}[enumerate]}{\end{easylist}}
\newenvironment{ilist}{\begin{easylist}[itemize]}{\end{easylist}}

\newcommand{\myspecial}[1]{\mathrm{#1}}

% imaginary unit
\newcommand{\iu}{{i\mkern1mu}}


\newcommand{\ipcdot}{\ip{\cdot}{\cdot}}
\newcommand{\iip}[2]{[#1,#2]}
\newcommand{\iipcdot}{\iip{\cdot}{\cdot}}

\newcommand{\dsum}{\oplus}
\newcommand{\ddiff}{\ominus}
% indefinite direct sum
\newcommand{\idsum}{[+]}
\newcommand{\iddiff}{[-]}

\DeclarePairedDelimiter{\Vector}{\lparen}{\rparen}

\newcommand{\tit}{\textit}
\newcommand{\cls}{\overline}
\newcommand{\eps}{\varepsilon}


\newcommand{\argmin}{\operatornamewithlimits{argmin}}
\newcommand{\argmax}{\operatornamewithlimits{argmax}}

\renewcommand{\Re}{\operatorname{Re}}
\renewcommand{\Im}{\operatorname{Im}}
\renewcommand{\phi}{\varphi} % TODO is there a prettier way to do that

\newcommand{\eexp}[1]{e^{#1}}

\DeclareMathOperator\atanh{atanh}

% ???
% \newcommand{\abs}[1]{\left| #1 \right|}
% \newcommand{\norm}[1]{\left\lVert #1 \right\rVert}
\newcommand*\Eval[3]{\left.#1\right\rvert_{#2}^{#3}}

% \lstnewenvironment{snippet}[1][]%
% {
%    \noindent
%    \minipage{\linewidth} 
%    \vspace{0.5\baselineskip}
%    \lstset{basicstyle=\ttfamily\footnotesize,frame=single,#1}}
% {\endminipage}

%\theoremstyle{remark}
%\newtheorem{remark}[theorem]{Remark}
%\include{header}
%%%%%%%%%%%%%%%%%%%%%%%%%%%%%%%%%%%%%%%%%%%%%%%%%%%%%%%%%%%%%%%%%%%%%%%%%%%%%%%

\numberwithin{theorem}{chapter}        % Numbers theorems "x.y" where x
                                        % is the section number, y is the
                                        % theorem number

%\renewcommand{\thetheorem}{\arabic{chapter}.\arabic{theorem}}

%\makeatletter                          % This sequence of commands will
%\let\c@equation\c@theorem              % incorporate equation numbering
%\makeatother                           % into the theorem numbering scheme

%\renewcommand{\theenumi}{(\roman{enumi})}

%%%%%%%%%%%%%%%%%%%%%%%%%%%%%%%%%%%%%%%%%%%%%%%%%%%%%%%%%%%%%%%%%%%%%%%%%%%%%%

\binoppenalty=10000
\relpenalty=10000

\addbibresource{thesis.bib}

\begin{document}
 % \renewcommand{\thelstlisting}{\thesection.\arabic{lstlisting}}
% Begin the front matter as required by Rackham dissertation guidelines
\initializefrontsections

\pagestyle{title}

\begin{center}
Университет ИТМО

\vspace{2cm}

Естественно-научный факультет

Кафедра высшей математики

\vspace{3cm}

{\Large Герасимов Дмитрий Александрович}

\vspace{2cm}

\vbox{\LARGE\bfseries
Исследование полноты резонансных состояний оператора Шредингера для модели квантовых графов
}

\vspace{4cm}

{\Large Научный руководитель: доктор физ.-мат. наук, \\ профессор кафедры ВМ И.~Ю.~Попов}

\vspace{6cm}

Санкт-Петербург\\ 2016
\end{center}

\newpage

\setcounter{page}{3}
\pagestyle{plain}

\tableofcontents
%\listoffigures

% Chapters
\startthechapters
\startprefacepage

\todo{REWRITE}

% В микроэлектронике для изготовления интегральных схем в основном используются полевые транзисторы. Полевой транзистор — прибор, в простейшем случае состоящий из трех контактов:
% \begin{easylist}[itemize]
% # исток — контакт, на который подаются носители заряда;
% # сток — контакт, с которого уходят носители заряда;
% # затвор — контакт, напряжением на котором можно регулировать ток, идущий от истока к стоку.
% \end{easylist}
% Фактически, транзистор является «управляемым сопротивлением», то есть прибором, проводимость которого можно контролировать напряжением на затворе, что и позволяет использовать его для реализации логических элементов.

% К сожалению, исследовать аналитически квантомеханические системы очень сложно, к примеру, уже в простейшей модели одномерной прямоугольной квантовой ямы (англ. 1D finite potential well) для расчета собственных энергий и функций, необходимых для расчета коэффициента проводимости, нужно решать трансцендентные уравнения. Естественно, в двумерном и трехмерном случаях, уравнения решать тем более сложнее.

% В рамках решения этой проблемы разработан подход, позволяющий изменить исходную модель, получив ее аппроксимацию, которая допускает аналитическое решение. В работах \cite{popov1992extension, popov1992resonator, popov1993zero} этот подход был применен для граничного условия Неймана. Однако, физически обоснованным граничным условием является граничное условие Дирихле, которое еще не было исследовано в рамках этого подхода.

% Целью данной работы является исследование квантового волновода с граничным условием Дирихле, аналитического решения уравнения Шредингера для этого волновода методом аппроксимации моделью нулевого радиуса и расчет его спектральных характеристик.
\chapter{Предварительные сведения}
\label{chapter1}

\section{Уравнение Шредингера}
\todo{рассеиватель}

\todo{резонаторы}

\section{S-матрица}

\todo{картиночку с полюсами и нулями??}

В общем виде волновые функции, являющиеся решениями уравнения Шредингера слева и справа от резонатора имеют вид:
\[
\psi_L(x) = A \eexp{\iu k x} + B \eexp{-\iu k x}
\]
\[
\psi_R(x) = C \eexp{\iu k x} + D \eexp{-\iu k x}
\]

S-матрица (матрица рассеяния) выражает зависимость исходящей волны от входящей:
\[
\begin{pmatrix} B \\ C \end{pmatrix} = S \begin{pmatrix} A \\ D \end{pmatrix}
\]

\todo{свойства}

\todo{симметричный/несимметричный случай}

\section{Преобразование Кэли}

Прямое преобразование Кэли (англ. Cayley transform) конформно отображает верхнюю комплексную полуплоскость (англ. upper half-plane) $\bbH = \{ x + \iu y \mid y > 0, x, y \in \bbR \}$ в комплексный единичный круг (англ. unit disk) $\bbD = \{ z \mid \abs{z} < 1 \}$:
\begin{equation}\label{eq:cayley}
W(z) = \frac{z - \iu}{z + \iu}
\end{equation}
, обратное преобразование Кэли аналогично отображает $\bbD$ в $\bbH$:
\begin{equation}\label{eq:cayley_inverse}
w(\zeta) = \iu \frac{1 + \zeta}{1 - \zeta}
\end{equation}

Важным свойством преобразование Кэли является инъективное отображение $\bbR$ в единичную окружность $\bbT = \partial \bbD =  \{z \mid \abs{z} = 1 \}$

\todo{перенести то что дальше куда-нибудь?}
Так как преобразование Кэли является трансформацией Мебиуса, оно сохраняет окружности. В частности, окружность с радиусом $r$ в нуле, под действием обратного преобразование Кэли перейдет в окружность с центром $C(r)$ и радиусом $R(r)$, где:

\begin{equation}\label{eq:c_and_r}
\begin{aligned}
   C(r) &= \Im \frac{w(r) + w(-r)}{2}
\\ R(r) &= \Im \frac{w(r) - w(-r)}{2}
\end{aligned}
\end{equation}

Легко заметить, что при стремлении $r$ к $1$, $R(r)$ стремится к бесконечности, а $C(r)$ стремится к $R(r)$, что естественно так как в пределе комплексная единичная окнужность отображается в вещественную ось.

\todo{написать про нотацию $z$ vs $\zeta$}

\todo{нотация, может сделать секцию?}

\todo{использовать шаблоны определений, теоремы и т.п.}

\section{Внутренние функции и произведение Бляшке}
Для скалярных внутренних функций $\phi$ определенных на комплексном единичном диске $\bbD$, существует критерий отстутствия сингулярного множителя \cite[стр. 99]{nikolskii}:
\[
\lim\limits_{r = 1} \int\limits_{\bbT} \log \abs{\phi(r \zeta)} d m(\zeta) = 0
\]

Так как $\det S$ является скалярной внутренней функцией, мы можем воспользоваться данным критерием для исследования полноты системы резонантных состояний.

\todo{мутно}

\todo{Неравенство Коши-Шварца??}

\section{Фиксирование нотации}
\subsection{Различные обозначения}

\begin{ilist}
# Жирным обозначаются вектора: к примеру, $\vb{r} \in \bbR^n$, без выделения жирным — их длины: $r = |\vb{r}|$;
# Сопряжение комплексных чисел обозначается как $\cconj{c}$;
# Сопряжение операторов обозначается как $A^*$.
\end{ilist}

\subsection{Скалярное произведение}

В данной работе используется \textbf{физическая} нотация. В частности, это означает, что скалярное произведение в $L^2(E)$ определено как $\ip{f}{g} = \int\limits_E \cconj{f(\vb{x})} g(\vb{x}) \dd \vb{x}$.

\section{Терминология теории рассеяния}
Состояния рассеяния (англ. scattering states) — решения уравнения Шредингера, соответствующие непрерывному спектру, и не лежащие в $L^2$. Также частно называется каналом рассеяния (англ. scattering channel).

Мода (англ. mode) — связанная часть состояния рассеяния. О них обычно говорят в контексте волноводов, имеющих двумерную или трехмерную геометрию, и допускающих только связанные состояния в поперечном направлении. К примеру, в волноводе с конфигурацией $\Omega = [-\infty, \infty] \times [0, H]$ допустимыми поперечными модами являются $\psi_n(y) = \sqrt{\frac{2}{H}} \sin(\frac{\pi n}{H} y)$, $n \in \bbN^+$. Если поперечной частью волны является мода $n$, говорят, что «волна распространяется на $n$-й моде».

Открытый канал (англ. open channel) — канал рассеяния, на котором волна потенциально может распространяться при данной энергии $E$. Закрытый канал (англ. closed channel) — канал рассеяния, на котором волна не может распространяться при данной энергии $E$. Для вышеупомянутого примера двумерного волновода при любой энергии $E$ всегда будут открыты каналы $\{n \mid n \in \bbN^+, \left(\frac{\pi n}{H}\right)^2 < E \}$, которых, очевидно, конечное число.

\section{Критерий сходимости}

Так как $\det S$ является внутренней функцией \todo{REF}, \todo{блаблабла} критерий полноты в пространстве единичного диска выглядит как

\begin{equation}\label{eq:crit_cayley}
\lim\limits_{r = 1} \int\limits_{\abs{\zeta} = r} \log \abs{\det S(\zeta)} d \zeta = 0
\end{equation}

Так как $S$-матрица естественным образом определеня на комплексной плоскости, иногда бывает удобнее работать с этим признаком в верхней полуплоскости $\bbH$, а не на единичном диске. Заменим в (\ref{eq:crit_cayley}) переменную, применив к интегралу преобразование Кэли (\ref{eq:cayley}) к дифференциалу и области интегрирования, получив:

\begin{equation}\label{eq:crit}
\lim\limits_{r \to 1} \int\limits_{C_r} \ln \abs{\det S(k)} \frac{2 \iu}{(k + i)^2} dk = 0
\end{equation}
, где $C_r$ — образ $\abs{\zeta} = r$ относительно обратного преобразования Кэли. Для удобства вычислений параметризуем $C_r$ как $C_r = \{R(r) \eexp{\iu t} + \iu C(r) \mid t \in [0, 2 \pi)\}$ (см. \ref{eq:c_and_r}). Для краткости, обозначим:

\[
s(k) = \abs{\det S(k)}
\]
, и после откидывания констант которые не влияют на сходимость/расходимость интеграла, получаем финальную форму критерия, которй и будем пользоваться в дальнейшем:

\begin{equation}\label{eq:critp}
\lim\limits_{r \to 1} \int\limits_{0}^{2 \pi} \ln s(R(r) \eexp{\iu t} + \iu C(r)) \frac{R}{(R(r) \eexp{\iu t} + \iu C(r) + i)^2} dt = 0
\end{equation}

\todo{$C + R = \Im w(r) = \frac{1 + r}{1 - r}$, $C - R = \Im w(-r) = \frac{1 - r}{1 + r}$} % TODO rename to 'survey'?
\chapter{Описание реализованного подхода}
\label{chapter2}

\section{Существующие результаты}
\todo{отдельная часть?}

\begin{figure}[!htb]
\begin{tikzpicture}[scale=0.8]
\input{pic/resonator_segment.tex}
\end{tikzpicture}
\caption{Одномерный резонатор-отрезок с $\delta$-образным барьером в $V$}
\end{figure}
\todo{тут на самом деле про дельту вообще не было доказано, может ее убрать?}
\todo{ссылка на работу}

\begin{figure}[!htb]
\centering
\begin{tikzpicture}[scale=1.1]
\newcommand{\Wglen}{6.0}; % waveguide length
\newcommand{\Warrlen}{2.0}; % wave arrow length

\coordinate (LLL) at (-\Wglen - 1, 0);
\coordinate (RRR) at ( \Wglen + 1, 0);
\coordinate (LL)  at (-\Wglen, 0);
\coordinate (RR)  at ( \Wglen, 0);
% change these to take resonator length into account
\coordinate (L)   at (-2, 0);
\coordinate (R)   at (2, 0);
%

\coordinate (U) at (0, 3); % upper point of resonator

% waveguide
\draw[ultra thick, dotted] (LLL) -- (LL);
\draw[ultra thick] (LL)--(L);
\draw[ultra thick] (R)--(RR);
\draw[ultra thick, dotted] (RR) -- (RRR);
%

\draw[<-] (-\Wglen + 1, 1.5) -- (-\Wglen + 1 + \Warrlen, 1.5) node [midway, above] {$R e^{-\iu k x}$};
\draw[->] (-\Wglen + 1, 0.5) -- (-\Wglen + 1 + \Warrlen, 0.5) node [midway, above] {$e^{\iu k x}$};
\draw[->] (\Wglen - 1 - \Warrlen, 0.5) -- (\Wglen - 1, 0.5)   node [midway, above] {$T e^{\iu k x}$};

\draw[ultra thick] (0, 2) node [above] {$\Omega$};

\draw (-2,0) arc (180:360:2cm and 0.5cm);
\draw[dashed] (-2,0) arc (180:0:2cm and 0.5cm);
\draw (0,2) arc (90:270:0.5cm and 2cm);
\draw[dashed] (0,2) arc (90:-90:0.5cm and 2cm);
\draw (0,0) circle (2cm);
\shade[ball color=blue!10!white,opacity=0.20] (0,0) circle (2cm);
\end{tikzpicture}
\caption{Трехмерный резонатор (квантовая точка)}
\end{figure}
\todo{ссылка на работу}
\todo{Переделать другие картинки чтобы они тоже были побольше}


\section{Модель резонатора типа «пучок»} % MINOR TODO пучок? =/

\todo{описание}

\todo{figure}

\todo{Определитель $S$-матрицы в замкнутой форме}



\section{Модель резонатора типа «кольцо»}
\subsection{Модель исследования}
% TODO http://tex.stackexchange.com/a/155317/5966 ?
\begin{figure}[!htb] % TODO maybe just ref instead?
\centering
\begin{tikzpicture}[scale=1.1] % TODO SCALE!!!
\newcommand{\Wglen}{6.0}; % waveguide length
\newcommand{\Warrlen}{2.0}; % wave arrow length
\newcommand{\Reslen}{0.0}; % resonator length

\coordinate (LLL) at (-\Wglen - 1, 0);
\coordinate (RRR) at ( \Wglen + 1, 0);
\coordinate (LL)  at (-\Wglen, 0);
\coordinate (RR)  at ( \Wglen, 0);
\coordinate (L)   at (-\Reslen, 0);
\coordinate (R)   at ( \Reslen, 0);
%

\coordinate (U) at (0, 3); % upper point of resonator

% waveguide
\draw[ultra thick, dotted] (LLL) -- (LL);
\draw[ultra thick] (LL)--(L);
\draw[ultra thick] (R)--(RR);
\draw[ultra thick, dotted] (RR) -- (RRR);
%

\draw[<-, thick] (-\Wglen, 1.5) -- (-\Wglen + \Warrlen, 1.5) node [midway, above] {\large $R e^{-\iu k x}$};
\draw[->, thick] (-\Wglen, 0.8) -- (-\Wglen + \Warrlen, 0.8) node [midway, above] {\large $e^{\iu k x}$};
\draw[->, thick] (\Wglen - \Warrlen, 0.5) -- (\Wglen, 0.5)   node [midway, above] {\large $T e^{\iu k x}$};

\draw (LL) node [below] {\large $\Omega_L$};
\draw (RR) node [below] {\large $\Omega_R$};

\draw[ultra thick] (0, 2) node [above] {\Large $\Omega$};

\draw[ultra thick] node [below] {\large $V$} (0, 2) circle[radius=2];
\end{tikzpicture}
\caption{Квантовый граф $\Gamma$, состоящий из полубесконечных ребер $\Omega_L, \Omega_R$ и рассеивателя $\Omega$, представляющего из себя окружность длиной $1$. В вершине $V$ уствновлена $\delta$-образная потенциальная яма глубиной $a$.}
\end{figure}

Мы рассматриваем случай рассеяния волны c волновым вектором $k$, приходящей слева направо. Таким образом, волновые функции на различных частях графа принимают следующий вид:

\begin{equation}\label{eq:ring_system}
\begin{aligned}
\psi_L(x) &= \eexp{\iu k x} + R \eexp{-\iu k x} \\
\psi_R(x) &= T \eexp{\iu k x}\\
\psi_\Omega(x) &= P \sin(k x) + Q \cos(k x)
\end{aligned}
\end{equation}
, где $R$ и $T$ — коэффициенты отражения и прохождения волны. Так как граф симметричен, его матрица рассеяния принимает вид
$S(k) = \begin{pmatrix} R(k) & T(k) \\ T(k) & R(k) \end{pmatrix}$.

В вершине $V$ мы ставим $\delta$-образную потенциальную яму высотой $a$, которая порождает следущие граничные условия:

\begin{equation}\label{eq:ring_bc}
\begin{aligned}
\psi_L(0) = \psi_R(0) = \psi_\Omega(0) = \psi_\Omega(1) \\ 
-\psi'_L(0) + \psi'_\Omega(0) - \psi'_\Omega(1) + \psi'_R(0) = a \psi_L(0)
\end{aligned}
\end{equation}


\subsection{Вычисление S-матрицы}
Определим коэффициенты прохождения и отражения, подставив функции (\ref{eq:ring_system}) в (\ref{eq:ring_bc}) и решив систему линейных уравнений:
\begin{align*}
& 1 + R &= T \\
& 1 + R &= Q \\
& Q \cos k + P \sin k &= T \\
& -P k \cos k + Q k \sin k + P k + \iu R k + \iu T k - \iu k &= T a
\end{align*}

Решив систему, получаем:

\begin{align*}
R(k) = -\frac{2 \, k \cos\left(k\right) + a \sin\left(k\right) - 2 \, k}{2 \, k \cos\left(k\right) + {\left(a - 2 i \, k\right)} \sin\left(k\right) - 2 \, k} \\
T(k) = -\frac{2 i \, k \sin\left(k\right)}{2 \, k \cos\left(k\right) + {\left(a - 2 i \, k\right)} \sin\left(k\right) - 2 \, k}
\end{align*}
, подставляя полученные значения коэффициента прохождения и отражения в S-матрицу, получаем определитель S-матрицы в замкнутой форме:

\begin{equation}\label{eq:ring_detS}
\det S = 
\frac
{\cos\left(k\right) + {\left(\frac{a}{2 k} + i\right)} \sin\left(k\right) - 1}
{\cos\left(k\right) + {\left(\frac{a}{2 k} - i\right)} \sin\left(k\right) - 1}
\end{equation}


\subsection{Доказательство неполноты резонансных состояний при $a=0$}
Рассмотрим случай $a=0$:
\[
\det S
= \frac
{\cos\left(k\right) + \iu \sin\left(k\right) - 1}
{\cos\left(k\right) - \iu \sin\left(k\right) - 1}
= \frac{\eexp{\iu k} - 1}{\eexp{-\iu k} - 1}
= -\eexp{i k}
\]

Из выражения выше легко заметить, что подынтегральное выражение в критерии полноты (\ref{eq:crit_cayley}) сводится к $\ln \abs{\det S} = \ln \eexp{- \Im k} = -\Im k$. Вычислим интеграл в пространстве единичного диска, для этого применим обратное преобразование Кэли (\ref{eq:cayley_inverse}), к подынтегральной функции: $\Im k \to \Im \left( \iu \frac{1 + \zeta}{1 - \zeta} \right) $.

\[
  \lim\limits_{r = 1} \int\limits_{\abs{\zeta} = r} \ln \abs{\det S(\zeta)} d \zeta
= \lim\limits_{r = 1} \int\limits_{\abs{\zeta} = r} \Im \left( \iu \frac{1 + \zeta}{1 - \zeta} \right)  d\zeta = \dots
\]
, параметризуем контур интегрирования полярными координатами: $\zeta \to r \eexp{\iu \phi}, d\zeta \to r \iu \eexp{\iu \phi}$:
\[
\dots = \lim\limits_{r = 1} \int\limits_{\abs{\zeta} = r} \Im \left( \iu \frac{1 + r \eexp{\iu \phi}}{1 - r \eexp{\iu \phi}} \right) r \iu \eexp{\iu \phi} d\phi
\]

Комплексный интеграл складывается из сумм интегралов действительной и мнимной части подынтегрального выражения. Рассчитаем мнимую часть:

\begin{align*}
   \Im \left(  \Im \left( \iu \frac{1 + r \eexp{\iu \phi}}{1 - r \eexp{\iu \phi}} \right) r \iu \eexp{\iu \phi} \right)
\\ &= r \Re \left(  \Re \left( \frac{1 + r \eexp{\iu \phi}}{1 - r \eexp{\iu \phi}} \right) \eexp{\iu \phi} \right)
\\ &= r \Re \left( \frac{1 + r \eexp{\iu \phi}}{1 - r \eexp{\iu \phi}} \right) \Re \left(   \eexp{\iu \phi} \right)
\\ &= r \Re \left( \frac{(1 + r \eexp{\iu \phi}) (1 - r \eexp{-\iu \phi}) }{(1 - r \eexp{\iu \phi}) (1 - r \eexp{-\iu \phi})} \right) \cos \phi
\\ &= r \Re \left( \frac{1 - r^2 + 2 \iu r \sin \phi}{1 + r^2 - 2 r \cos \phi} \right) \cos \phi
\\ &= r \frac{1 - r^2}{1 + r^2 - 2 r \cos \phi} \cos \phi 
% \\ &= \frac{1 - R^2}{2} \frac{\cos \phi}{\frac{1 + R^2}{2R} - \cos \phi} 
\end{align*}

% MINOR доинтегрировать

Интегрируя, получаем $2 \pi r^2$, что значит, что предел мнимой части при $r \to 1$ равен $2 \pi$, следовательно, по критерию полноты, система резонатных состояний графа $\Gamma$ не является полной на кольце $\Omega$.

\mtodo{график S-матрицы?}


\subsection{Исследование полноты при $a \ne 0$}

\todo{Результаты численного интегрирования}

\todo{Обоснование, почему сходится}

% % \ref{eq:ring_detS}
% Оценим снизу выражение $\abs{\det S(k)}$, для этого отдельно оценим снизу модуль числителя и сверху модуль знаменателя:

% \begin{align*}
%        & \abs{\cos\left(k\right) + {\left(\frac{a}{2 k} - i\right)} \sin\left(k\right) - 1}
% \\ =   & \abs{\eexp{-\iu k}  + \frac{a}{2 k} \frac{\eexp{\iu k} + \eexp{-\iu k}}{2 \iu} - 1}
% \\ \le & \abs{\eexp{-\iu k}}  + \abs{\frac{a}{2 k} \frac{\eexp{\iu k} + \eexp{-\iu k}}{2 \iu}} + 1
% \\ \le & \eexp{y} + \frac{a}{4 \sqrt{x^2 + y^2} } (\eexp{y} + \eexp{-y}) + 1
% \end{align*}


% \begin{align*}
%        & \abs{\cos\left(k\right) + {\left(\frac{a}{2 k} + i\right)} \sin\left(k\right) - 1}
% \\ =   & \abs{\eexp{\iu k}  + \frac{a}{2 k} \frac{\eexp{\iu k} + \eexp{-\iu k}}{2 \iu} - 1}
% \\ =   & \abs{\eexp{\iu k} (1 + \frac{a}{4 \iu k}) + \frac{a}{4 \iu k} \eexp{-\iu k} - 1}
% \\ =   & \abs{\eexp{\iu x} \eexp{-y}  (1 + \frac{a}{4 \iu k}) + \frac{a}{4 \iu k} \eexp{-\iu x} \eexp{y} - 1}
% % \\ \le & \abs{\eexp{-\iu k}  + \frac{a}{2 k} \sin k - 1}
% % \\ \le & \abs{\eexp{-\iu k}}  + \abs{\frac{a}{2 k} \sin k} + 1
% % \\ \le & \eexp{y} + \abs{\frac{a}{2 k} \frac{\eexp{\iu k} + \eexp{-\iu k}}{2 \iu}} + 1
% % \\ \le & \eexp{y} + \frac{a}{4 \sqrt{x^2 + y^2} } (\eexp{y} + \eexp{-y}) + 1
% \end{align*}

% Оценим выражение снизу модулем его мнимой части:

% \begin{align*}
% \Im &= \sin x \  \eexp{-y} (1 + \Re \frac{a}{4 \iu (x + \iu y)}) + \sin (-x) \eexp{y} \Re \frac{a}{4 \iu (x + \iu y)}
% \\  &= \sin x \  \eexp{-y} (1 - \frac{a y}{4 (x^2 + y^2)}) + \sin x \ \eexp{y} \frac{a y}{4 (x^2 + y^2)}
% \end{align*}

% \begin{align*}
% \Re &= \cos x \  \eexp{-y} (1 + \Re \frac{a}{4 \iu (x + \iu y)}) + \cos (-x) \eexp{y} \Re \frac{a}{4 \iu (x + \iu y)} - 1
% \\  &= \cos x \  \eexp{-y} (1 - \frac{a y}{4 (x^2 + y^2)}) + \cos x \ \eexp{y} \frac{a y}{4 (x^2 + y^2)} - 1
% \end{align*}


% \det S = 
% \frac
% {\cos\left(k\right) + {\left(\frac{a}{2 k} + i\right)} \sin\left(k\right) - 1}
% {\cos\left(k\right) + {\left(\frac{a}{2 k} - i\right)} \sin\left(k\right) - 1}
% \end{equation}


\todo{Доказать!}

\todo{Объяснить результат}

\todo{другие виды резонаторов?}

\section{Модель резонатора типа «решетка»}

\todo{????, показать результаты численного интегрирования? Не успею, наверное, аналитически. Можно хотя бы S-матрицу и помахать руками?} % TODO rename to 'description'?
\chapter{Результаты} 
\label{chapter3}

\todo{REWRITE}
 % TODO rename to 'results'?
\startconclusionpage

\todo{REWRITE}

% В работе предложен метод получения приближенных спектральных и проводящих характеристик двумерного волновода сложной структуры, основанный на самосопряженных расширениях симметрических операторов и выходе в понтрягинское пространство функций. Для моделей подобных конфигураций не существует аналитических решений, способы численного решения автору также неизвестны.

% Данная работа отличается новизной, так как ранее подобные вычисления были проделаны только для волноводов с граничным условием Неймана \cite{popov1993zero}, которое, во-первых, не является физически обоснованным, а во-вторых, значительно проще для вычислений в рамках теории расширений симметрических операторов в том смысле, что расчеты могут быть проделаны в пространстве $\mcL^2$.

% На основе полученных результатов сделан вывод, что данная модель подходит в качестве возможной реализации наноразмерных транзисторов, непосредственно использующих квантовые эффекты. Построена зависимость коэффициента прохождения от энергии входящей волны и обнаружены резонансы. Построена зависимость коэффициента прохождения от геометрии резонатора, на которой также обнаруживается резонанс, что позволяет использовать данную нелинейность в коэффициенте прохождения для реализации транзистора.

% Результаты данной работы также могут быть использованы в расчетах расширений для более сложных квантомеханических операторов, в частности, оператора Дирака, учитывающего релятивистские эффекты, и более сложных моделей, например, трехмерных волноводов.

% \todo{Список литераторы: сделать P. вместо С.}
% \chapter{Trash}
\label{trash}

\subsection{Дефектные элементы}
Пусть направление нормали к границе между областями совпадает с осью $y$, тогда: дефектный элемент

\subsection{Волновод}



\subsection{Цилиндрические координаты}

\[
\vb{r} = (r, \theta, z)
\]


\subsection{Волновод}


\subsection{Преобразование Фурье}
В задачах квантовой механики часто удобно и более естественно производить анализ в импульсном представлении, нежели чем в координатном. Они связаны между собой преобразованием Фурье (напоминаем, что пользуемся АСЕ, засчет чего константа $\hbar$ опускается):

\subsubsection{Из координатного представления в моментное}
Пусть имеется волновая функция $\psi: \bbR^n \to \bbC$. 

\subsubsection{Из импульсного представления в координатное}


\printbibliography

%\startappendices
%\input{parts/appendix1.tex}

\end{document}
